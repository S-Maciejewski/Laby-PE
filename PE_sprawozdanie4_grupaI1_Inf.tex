\documentclass[polish,a4paper]{article}
\usepackage{amsmath}
\usepackage{amssymb, amsfonts, amsthm, amsmath, bm}
\usepackage[T1]{fontenc}
\usepackage[utf8]{inputenc}
\usepackage{babel}
\usepackage{pslatex}
\usepackage{pgfplots}
\usepackage{hhline}
\usepackage[american]{circuitikz} 
\usepackage{anysize}
\usepackage{graphicx}
\DeclareGraphicsExtensions{.jpg}
\marginsize{2.5cm}{2.5cm}{3cm}{3cm}
\bibliographystyle{IEEEtran}


%makro do indeksów w tabeli
\newcommand{\PRzFieldDsc}[1]{\sffamily\bfseries\scriptsize #1}

%makro do informacji w tabeli
\newcommand{\PRzFieldCnt}[1]{\itshape #1}

%potężne makro tworzące tabelę z informacjami o teamie
\newcommand{\PRzHeading}[8]{
%% #1 - nazwa laboratorium
%% #2 - kierunek 
%% #3 - specjalność 
%% #4 - rok studiów 
%% #5 - symbol grupy lab.
%% #6 - temat 
%% #7 - numer lab.
%% #8 - skład grupy ćwiczeniowej

\begin{center}
\begin{tabular}{ p{0.32\textwidth} p{0.15\textwidth} p{0.15\textwidth} p{0.12\textwidth} p{0.12\textwidth} }

  &   &   &   &   \\
\hline
\multicolumn{5}{|c|}{}\\[-1ex]
\multicolumn{5}{|c|}{{\LARGE #1}}\\
\multicolumn{5}{|c|}{}\\[-1ex]

\hline
\multicolumn{1}{|l|}{\PRzFieldDsc{Kierunek}}	& \multicolumn{1}{|l|}{\PRzFieldDsc{Specjalność}}	& \multicolumn{1}{|l|}{\PRzFieldDsc{Rok studiów}}	& \multicolumn{2}{|l|}{\PRzFieldDsc{Symbol grupy lab.}} \\
\multicolumn{1}{|c|}{\PRzFieldCnt{#2}}		& \multicolumn{1}{|c|}{\PRzFieldCnt{#3}}		& \multicolumn{1}{|c|}{\PRzFieldCnt{#4}}		& \multicolumn{2}{|c|}{\PRzFieldCnt{#5}} \\

\hline
\multicolumn{4}{|l|}{\PRzFieldDsc{Temat Laboratorium}}		& \multicolumn{1}{|l|}{\PRzFieldDsc{Numer lab.}} \\
\multicolumn{4}{|c|}{\PRzFieldCnt{#6}}				& \multicolumn{1}{|c|}{\PRzFieldCnt{#7}} \\

\hline
\multicolumn{5}{|l|}{\PRzFieldDsc{Skład grupy ćwiczeniowej oraz numery indeksów}}\\
\multicolumn{5}{|c|}{\PRzFieldCnt{#8}}\\

\hline
\multicolumn{3}{|l|}{\PRzFieldDsc{Uwagi}}	& \multicolumn{2}{|l|}{\PRzFieldDsc{Ocena}} \\
\multicolumn{3}{|c|}{\PRzFieldCnt{\ }}		& \multicolumn{2}{|c|}{\PRzFieldCnt{\ }} \\

\hline
\end{tabular}
\end{center}
}
%koniec potężnego makro do tabeli

\begin{document}

%stworzenie tabeli - miejsce na zmienianie danych w tabeli
%indeksy do uzupełnienia
\PRzHeading{Laboratorium Podstaw Elektroniki}{Informatyka}{--}{I}{I1}{Rezonans w obwodach RLC}{3}{Ewa Fengler(132219), Sebastian Maciejewski(132275), Jan Techner(132332)}{}

%ZADANIA

%\section*{Cel}

\section{Zadanie 1.2}

%1:  Zapoznaj się ze schematem (rys. 4) zestawu pomiarowego

2.
Rzeczywiste wartości rezystancji wykorzystanych elementów:
%bez diody - nie jest potrzebna

\begin{center}
\begin{tabular}{|c||c|c|c|}
\hline
\textbf{Element} & \textbf{Wartość zadana} & \textbf{Oznaczenie} & \textbf{Wartość zmierzona}\\
\hhline{|=#=|=|=|}
\textbf{R1} & 1k$\Omega$ & brązowy, czarny, czerwony, złoty & $984,3\Omega\pm5\%$\\
\hline
\textbf{R2} & 3M$\Omega$ & pomarańczowy, czarny, zielony, złoty & 3,009M$\Omega\pm5\%$\\
\hline
\end{tabular}
\end{center}

%3.Przygotuj zestaw pomiarowy według rysunku 4.a

4.
[opis] 

\begin{center}
\begin{tabular}{|c|c||c|c|}
\hline
\boldsymbol{$U_z$} [V] & \boldsymbol{$U_R$} [V] & \boldsymbol{$U_d$} [V]& \boldsymbol{$I_d$} [mA]\\
\hhline{|=|=#=|=|}
0 & 0 &&\\ \hline
0,2 & 0,005 &&\\ \hline
0,4 & 0,11&&\\ \hline
0,6 & 0,274&&\\ \hline
0,8 & 0,424&&\\ \hline
1 & 0,575&&\\ \hline
1,5 & 1,06&&\\ \hline
2 & 1,57&&\\ \hline
2,5 & 2,09&&\\ \hline
3 & 2,61&&\\ \hline
3,5 & 3,03&&\\ \hline
4 & 3,51&&\\ \hline
4,5 & 4,03&&\\ \hline
5 & 4,55&&\\ \hline
\hline
\end{tabular}
\end{center}

%5. Zmodyfikuj zestaw pomiarowy według schematu z rysunku 4.b

6.
[opis]

\begin{center}
\begin{tabular}{|c|c||c|c|}
\hline
\boldsymbol{$U_z$} [V] & \boldsymbol{$U_R$} [mV] & \boldsymbol{$U_d$} [V]& \boldsymbol{$I_d$} [mA]\\
\hhline{|=|=#=|=|}
0 & 0 &&\\ \hline
5 & 3,35 &&\\ \hline
10 & 3,94 &&\\ \hline
15 & 4,52 &&\\ \hline
20 & 4,75 &&\\ \hline
\hline
\end{tabular}
\end{center}

%7, 8. - uzupełnione tabelki: U_d = U_z - U_R, I_d = U_R/R
%9.  Na wspólnym wykresie zobrazuj przebieg charakterystyki Id = f(Ud) dla diody spolaryzowanej w kierunku zaporowym i przewodzenia

\section{Zadanie 1.3}



\bibliography{IEEEabrv,refs}

\begin{thebibliography}{9}

\bibitem{rlc}
  W trakcie przeprowadzania doświadczeń i pisania sprawozdania zespół korzystał głównie z materiałów ze strony http://mariusznaumowicz.ddns.net/materialy.html oraz z wiedzy własnej.

\end{thebibliography}

\end{document}7