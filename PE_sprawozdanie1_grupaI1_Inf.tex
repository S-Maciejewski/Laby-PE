\documentclass[polish,a4paper]{article}
\usepackage[T1]{fontenc}
\usepackage[utf8]{inputenc}
\usepackage{babel}
\usepackage{pslatex}
\usepackage[european]{circuitikz} 
\usepackage{anysize}
\marginsize{2.5cm}{2.5cm}{3cm}{3cm}

\newcommand{\PRzFieldDsc}[1]{\sffamily\bfseries\scriptsize #1}

\newcommand{\PRzFieldCnt}[1]{\itshape #1}

\newcommand{\PRzHeading}[8]{
%% #1 - nazwa laboratorium
%% #2 - kierunek 
%% #3 - specjalność 
%% #4 - rok studiów 
%% #5 - symbol grupy lab.
%% #6 - temat 
%% #7 - numer lab.
%% #8 - skład grupy ćwiczeniowej

\begin{center}
\begin{tabular}{ p{0.32\textwidth} p{0.15\textwidth} p{0.15\textwidth} p{0.12\textwidth} p{0.12\textwidth} }

  &   &   &   &   \\
\hline
\multicolumn{5}{|c|}{}\\[-1ex]
\multicolumn{5}{|c|}{{\LARGE #1}}\\
\multicolumn{5}{|c|}{}\\[-1ex]

\hline
\multicolumn{1}{|l|}{\PRzFieldDsc{Kierunek}}	& \multicolumn{1}{|l|}{\PRzFieldDsc{Specjalność}}	& \multicolumn{1}{|l|}{\PRzFieldDsc{Rok studiów}}	& \multicolumn{2}{|l|}{\PRzFieldDsc{Symbol grupy lab.}} \\
\multicolumn{1}{|c|}{\PRzFieldCnt{#2}}		& \multicolumn{1}{|c|}{\PRzFieldCnt{#3}}		& \multicolumn{1}{|c|}{\PRzFieldCnt{#4}}		& \multicolumn{2}{|c|}{\PRzFieldCnt{#5}} \\

\hline
\multicolumn{4}{|l|}{\PRzFieldDsc{Temat Laboratorium}}		& \multicolumn{1}{|l|}{\PRzFieldDsc{Numer lab.}} \\
\multicolumn{4}{|c|}{\PRzFieldCnt{#6}}				& \multicolumn{1}{|c|}{\PRzFieldCnt{#7}} \\

\hline
\multicolumn{5}{|l|}{\PRzFieldDsc{Skład grupy ćwiczeniowej oraz numery indeksów}}\\
\multicolumn{5}{|c|}{\PRzFieldCnt{#8}}\\

\hline
\multicolumn{3}{|l|}{\PRzFieldDsc{Uwagi}}	& \multicolumn{2}{|l|}{\PRzFieldDsc{Ocena}} \\
\multicolumn{3}{|c|}{\PRzFieldCnt{\ }}		& \multicolumn{2}{|c|}{\PRzFieldCnt{\ }} \\

\hline
\end{tabular}
\end{center}
}

\begin{document}

\PRzHeading{Laboratorium Podstaw Elektroniki}{Informatyka}{--}{I}{I1}{Wprowadzenie}{1}{Ewa Fengler(132219), Sebastian Maciejewski(132275), Jan Techner(132332)}{}

%ZADANIA

\section{Zadanie A}

\subsection*{Cel}

Ćwiczenie ma na celu zaznajomienie z podstawowymi wielkościami fizycznymi służącymi do opisu własności obwodów elektrycznych oraz oznaczeniami elementów tych obwodów (cewki, rezystory i kondensatory). Aby prawidłowo wykonać opisane w poleceniu pomiary konieczne jest także nauczenie się obsługi przyrządów pomiarowych - pomiar rezystancji, pojemności kondensatorów i pojemności cewek przy pomocy multimetru RIGOL DS1022.

\subsection{Część I}
Odczytanie wartości rezystancji na podstawie kodu paskowego rezystorów oraz pomiar jej wartości za pomocą multimetru.\\

\begin{center}
\begin{tabular}{|c|c|c|c|}
\hline
\textbf{R} & \textbf{Barwy} & \textbf{Odczyt} & \textbf{Pomiar}\\
\hline
R1 & czerwony, czerwony, brązowy & $220\Omega$ & $220\Omega$\\
\hline
R2 & pomarańczowy, pomarańczowy, zielony & $3,3M\Omega$ & $3,25M\Omega$\\
\hline
R3 & brązowy, czarny, brązowy & $100\Omega$ & $98\Omega$\\
\hline
R4 & brązowy, czarny, czerwony & $1k\Omega$ & $0,99k\Omega$\\
\hline
R5 & czerwony, czarny, czerwony & $2k\Omega$ & $1,95k\Omega$\\
\hline
R6 & czerwony, czarny, zielony & $2M\Omega$ & $1,97M\Omega$\\
\hline
\end{tabular}
\end{center}

\subsection{Część II}
Odczytanie pojemności kondensatorów oraz pomiar ich pojemności przy pomocy mostka pomiarowego.\\

\begin{center}
\begin{tabular}{|c|c|c|c|}
\hline
\textbf{C} & \textbf{Oznaczenie} & \textbf{Odczyt} & \textbf{Pomiar}\\
\hline
C1 & 223 & 22nF & 33,4nF\\
\hline
C2 & 10n & 10nF & 8,4nF\\
\hline
C3 & 132 & 3,3nF & 2,9nF\\
\hline
C4 & 222 & 2,2nF & 2,3nF\\
\hline
C5 & 10$\mu$F & 10$\mu$F & 10,7$\mu$F\\
\hline
C6 & 12$\mu$F & 22$\mu$F & 20,9$\mu$F\\
\hline
\end{tabular}
\end{center}

\subsection{Część III}
Pomiar indukcyjności wybranych cewek używając mostka pomiarowego.\\

\begin{center}
\begin{tabular}{|c|c|}
\hline
\textbf{L} & \textbf{Pomiar}\\
\hline
L1 & 30,08nH\\
\hline
L2 & 30,28nH\\
\hline
L3 & 30,9$\mu$H\\
\hline
\end{tabular}
\end{center}

\section{Zadanie B}

\subsection*{Cel}

Zadanie B ma na celu zapoznanie się z metodą obliczania oporu zastępczego dla rezystorów połączonych szeregowo i równolegle (także dla całych obwodów) i naukę umiejętności budowania oraz pomiaru właściwości obwodów na płytce prototypowej. W sposób naturalny ćwiczenie kształci również umiejętność odczytywania schematów obwodów.\\

\subsection{Część I}

Obliczenie i wyprowadzenie wzoru dla rezystancji zastępczej obwodu przedstawionego poniżej.\\

\begin{figure}[!h]
\centering
\begin{circuitikz}[scale=1.1, font = \scriptsize]
\draw (0,0) to [R, l=$R_7$, a=1$\Omega$] (2,0) -- (2, 0.7) to [R, l=$R_5$, a=100$\Omega$] (4,0.7) -- (4,0)
(2,0) -- (2,-0.7) to [R, l=$R_6$, a=200$\Omega$] (4,-0.7) --  (4,0) -- (4.5,0) -- (4.5,2.1) to [R, l=$R_1$, a=2k$\Omega$] (6.5,2.1) -- (6.5,-2.1) to [R, l_=$R_4$, a^=270$\Omega$] (4.5,-2.1) -- (4.5,0)
(4.5,0.7) to [R, l=$R_2$, a=3k$\Omega$] (6.5,0.7)
(4.5,-0.7) to [R, l=$R_3$, a=1k$\Omega$] (6.5,-0.7)
(6.5,0) -- (7,0) -- (7,0.7) to [R, l=$R_8$, a=1$\Omega$] (9, 0.7) -- (9, -0.7) to [R, l_=$R_9$, a^=100$\Omega$] (7,-0.7) -- (7,0)
(9,0) -- (10,0);
\draw (-0.2,0) node {A}
	  (10.2,0) node {B};  
\end{circuitikz}
\end{figure}

$$
R_{z} = R_{7} + \frac{1}{\frac{1}{R_{5}} + \frac{1}{R_{6}}} + \frac{1}{\frac{1}{R_{1}} + \frac{1}{R_{2}} + \frac{1}{R_{3}} + \frac{1}{R_{4}}} + \frac{1}{\frac{1}{R_{8}} + \frac{1}{R_{9}}} = $$ \\ $$R_{7} + \frac{R_{5} \cdot R_{6}}{R_{5} + R_{6}} + \frac{R_{1}\cdot R_{2}\cdot R_{3}\cdot R_{4}}{ R_{1}\cdot R_{2}\cdot R_{3} + R_{1}\cdot R_{2}\cdot R_{4} + R_{1}\cdot R_{3}\cdot R_{4} + R_{2}\cdot R_{3}\cdot R_{4}} + \frac{R_{8}\cdot R_{9}}{R_{8} + R_{9}}
$$

\begin{center}
$R_{z} = 1k\Omega + \frac{100\Omega\cdot200\Omega}{100\Omega + 200\Omega} + \frac{1\Omega\cdot100\Omega}{1\Omega + 100\Omega}+$\\
\vspace{0,5cm}
\begin{large}
$+ \frac{2k\Omega\cdot3k\Omega\cdot1k\Omega\cdot270\Omega}{2k\Omega\cdot3k\Omega\cdot1k\Omega + 2k\Omega\cdot3k\Omega\cdot270\Omega + 2k\Omega\cdot1k\Omega\cdot270\Omega + 3k\Omega\cdot1k\Omega\cdot270\Omega}$\\
\end{large}
\end{center}
$$
R_{z} = 1248,26\Omega
$$

\newpage
\subsection{Część II}
Obliczanie rezystancji obwodów oraz jej pomiar dla obwodu zbudowanego na płytce prototypowej.\\
Pod każdym przykładem zamieszczono porównanie obliczonej rezystancji z jej zmierzoną wartością.\\

\subsubsection{Obwód 1.}

\begin{figure}[!h]
\centering
\begin{circuitikz}[scale=1.1, font = \scriptsize]
\draw (0,0) -- (0.5,0) -- (0.5,0.7) to [R, l=$R_2$, a=2k$\Omega$] (2.5,0.7) to [R, l=$R_3$, a=2k$\Omega$] (4.5, 0.7) -- (4.5,0)
	  (0.5,0) -- (0.5,-0.7) to [R, l=$R_1$, a=1k$\Omega$] (4.5,-0.7) -- (4.5,0) -- (5,0); 
\path (-0.2,0) node {A}
	  (5.2,0) node {B};
\end{circuitikz}
\end{figure}

$$
R_{z} = \frac{1}{\frac{1}{R_{1}} + \frac{1}{R_{2} + R_{3}}} = \frac{(R_{2}+R_{3})\cdot R_{1}}{R_{1}+R_{2}+R_{3}}
$$
\\
$$
R_{z} = \frac{(2k\Omega+2k\Omega)\cdot 1k\Omega}{1k\Omega+2k\Omega+2k\Omega} = 800\Omega
$$
\\
\begin{tabular}{|c|c|}
\hline
\textbf{Obliczenia} & \textbf{Pomiar}\\
\hline
$800\Omega$ & $790\Omega$\\
\hline
\end{tabular}

\subsubsection{Obwód 2.}

\begin{figure}[!h]
\centering
\begin{circuitikz}[scale=1.1, font = \scriptsize]
\draw (0,0) -- (0.5,0) -- (0.5,-0.7) to [R, l=$R_4$, a=100$\Omega$] (7,-0.7) -- (7, 0)
	  (0.5,0) -- (0.5,1.4) -- (1,1.4) -- (1,2.1) to [R, l=$R_1$, a=1k$\Omega$] (3,2.1) -- (3,0.7)
	  (1,1.4) -- (1,0.7) to [R, l=$R_2$, a=2k$\Omega$] (3,0.7) -- (3,1.4) to [R, l=$R_3$, a=1k$\Omega$] (5,1.4) to [R, l=$R_5$, a=100$\Omega$] (7,1.4) -- (7,0)-- (7.5,0);
\draw (-0.2,0) node {A}
	  (7.7,0) node {B};
\end{circuitikz}
\end{figure}

$$
R_{z} = \frac{1}{\frac{1}{R_{4}} + \frac{1}{ \frac{1}{\frac{1}{R_{1}} + \frac{1}{R_{2}}}+R_{3} + R_{5}}} = \frac{R_{4}\cdot(R_{1}\cdot R_{2} + (R_{1} + R_{2})\cdot(R_{3}+R_{5}))}{R_{1}\cdot R_{2} + (R_{1}+R_{2})\cdot(R_{3}+R_{5}) + (R_{1}+R_{2})\cdot R_{4}}
$$
\\
$$
R_{z} = \frac{100\Omega\cdot(1k\Omega\cdot 2k\Omega + (1k\Omega + 2k\Omega)\cdot(1k\Omega+100\Omega)}{1k\Omega\cdot 2k\Omega + (1k\Omega+2k\Omega)\cdot(1k\Omega+100\Omega) + (1k\Omega+2k\Omega)\cdot 100\Omega} = 94,6 \Omega
$$
\\
\begin{tabular}{|c|c|}
\hline
\textbf{Obliczenia} & \textbf{Pomiar}\\
\hline
$94,6\Omega$ & $95\Omega$\\
\hline
\end{tabular}

\newpage
\subsubsection{Obwód 3.}

\begin{figure}[!h]
\centering
\begin{circuitikz}[scale=1.1, font = \scriptsize]
\draw (0,0) -- (1,0) to [R, l=$R_1$, a=2k$\Omega$] (1,-2) -- (0, -2)
	  (1,0) -- (2,0) to [R, l=$R_3$, a=100$\Omega$] (4,0)
	  (2,0) -- (2,-1.4) to [R, l=$R_4$, a=1k$\Omega$] (4,-1.4) -- (4,0) -- (5,0) to [R, l=$R_2$, a=2k$\Omega$] (5, -2) -- (6,-2)
	  (5,0) -- (6,0);
\path (-0.2,0) node {A}
	  (-0.2,-2) node {B};
\end{circuitikz}
\end{figure}

$$
R_{z} = R_{1} = 2k\Omega
$$
\\
\begin{tabular}{|c|c|}
\hline
\textbf{Obliczenia} & \textbf{Pomiar}\\
\hline
$2k\Omega$ & $1,952k\Omega$\\
\hline
\end{tabular}

\subsubsection{Obwód 4.}

\begin{figure}[!h]
\centering
\begin{circuitikz}[scale=1.1, font = \scriptsize]
\draw (0,0) -- (1,0) -- (1,-0.7) to [R, l=$R_4$, a=1k$\Omega$] (7.5,-0.7) -- (7.5, 0)
	  (1,0) -- (1,1.4) -- (1.5,1.4) -- (1.5,2.1) to [R, l=$R_5$, a=100$\Omega$] (3.5,2.1) to [R, l=$R_1$, a=1k$\Omega$] (5.5,2.1) -- (5.5,1.4)
	  (1.5,1.4) -- (1.5,0.7) to [R, l=$R_2$, a=2k$\Omega$] (5.5,0.7) -- (5.5,1.4)
	  (5.5,1.4) to [R, l=$R_3$, a=2k$\Omega$] (7.5,1.4) -- (7.5,0)--(8.5,0);
\path (-0.2,0) node {A}
	  (8.7,0) node {B};
\end{circuitikz}
\end{figure}

$$
R_{z} = \frac{1}{\frac{1}{R_{4}} + \frac{1}{ \frac{1}{\frac{1}{R_{2}} + \frac{1}{R_{5} + R_{1}}}+R_{3}}} = \frac{R_{4}\cdot ((R_{1} + R_{5})\cdot R_{2} + (R_{1} + R_{2} + R_{5})\cdot R_{3})}{(R_{1} + R_{2} + R_{5})\cdot (R_{3} + R_{4}) + (R_{1} + R_{5})\cdot R_{2}}
$$
\\
$$
R_{z} = \frac{1k\Omega\cdot ((1k\Omega + 100\Omega)\cdot 2k\Omega + (1k\Omega + 2k\Omega + 100\Omega)\cdot 2k\Omega)}{(1k\Omega + 2k\Omega + 100\Omega)\cdot (2k\Omega + 1k\Omega) + (1k\Omega + 100\Omega)\cdot 2k\Omega} = 730,4 \Omega
$$
\\
\begin{tabular}{|c|c|}
\hline
\textbf{Obliczenia} & \textbf{Pomiar}\\
\hline
$730,4\Omega$ & $725\Omega$\\
\hline
\end{tabular}

\newpage
\subsubsection{Obwód 5.}

\begin{figure}[!h]
\centering
\begin{circuitikz}[scale=1.1, font = \scriptsize]
\draw (0,0) -- (1,0) to [R, l=$R_1$, a=2k$\Omega$] (1,-2) -- (0, -2)
	  (1,0) -- (2,0) to [R, l=$R_3$, a=2k$\Omega$] (4,0)
	  (2,0) -- (2,-1.2) to [R, l=$R_4$, a=1k$\Omega$] (4,-1.2) -- (4,0) -- (5,0) to [R, l=$R_2$, a=2k$\Omega$] (5, -2) -- (6,-2)
	  (5,0) -- (6,0)
	  (1,-2) -- (5, -2);
\path (-0.2,0) node {A}
	  (-0.2,-2) node {B};
\end{circuitikz}
\end{figure}

$$
R_{z} = \frac{1}{\frac{1}{R_{1}} + \frac{1}{ \frac{1}{\frac{1}{R_{4}} + \frac{1}{R_{3}}}+R_{2}}} = \frac{R_{1}\cdot(R_{3}\cdot R_{4} + R_{2}\cdot(R_{3} + R_{4}))}{R_{3}\cdot R_{4} + (R_{1}\cdot R_{2})\cdot (R_{3} + R_{4})}
$$ 
\\
$$
R_{z} = \frac{2k\Omega\cdot(2k\Omega\cdot 1k\Omega + 2k\Omega\cdot(2k\Omega + 1k\Omega))}{2k\Omega\cdot 1k\Omega + (2k\Omega\cdot 2k\Omega)\cdot (2k\Omega + 1k\Omega)} = 1142,9\Omega
$$ 
\\
\begin{tabular}{|c|c|}
\hline
\textbf{Obliczenia} & \textbf{Pomiar}\\
\hline
$1142,9\Omega$ & $1119\Omega$\\
\hline
\end{tabular}


\subsubsection{Obwód 6.}

\begin{figure}[!h]
\centering
\begin{circuitikz}[scale=1.1, font = \scriptsize]
\draw (0,0) -- (0.5,0) to [R, l=$R_3$, a=1k$\Omega$] (2.5,0) -- (2.5, -0.7)
	  (0,-1.4) -- (0.5,-1.4) to [R, l=$R_4$, a=2k$\Omega$] (2.5,-1.4)--(2.5,-0.7) -- (3,-0.7) -- (3,0) to [R, l=$R_1$, a=100$\Omega$] (5,0) --(5,-0.7)
	  (3,-0.7) -- (3,-1.4) to [R, l=$R_2$, a=2k$\Omega$] (5,-1.4)-- (5,-0.7) -- (5.5, -0.7)
	  (0.5, -1.4) to (0.5, -2.8) to [R, l=$R_5$, a=1k$\Omega$] (5.5,-2.8)-- (5.5,-0.7);
\path (-0.2,0) node {A}
	  (-0.2,-1.4) node {B};
\end{circuitikz}
\end{figure}

W celu zwiększenia czytelności obliczeń, w tym przypadku wprowadzimy dodatkowe oznaczenia:
$$
R_{6} = \frac{R_{1}\cdot R_{2}}{R_{1}+R_{2}} = \frac{100\Omega\cdot 2k\Omega}{100\Omega+2k\Omega} = 95,24\Omega
$$
$$
R_{7} = R_{6} + R_{5} = 1000\Omega + 95,24\Omega = 1095,24\Omega
$$
$$
R_{8} = \frac{R_{4}\cdot R_{7}}{R_{4}+R_{7}} = \frac{2000\Omega\cdot 1095,24\Omega}{2000\Omega + 1095,24\Omega} = 707,7\Omega
$$
\\
$$
R_{z} = R_{3} + R_{8} = 1000\Omega + 707,7\Omega = 1707,7\Omega\\
$$
\\
\begin{tabular}{|c|c|}
\hline
\textbf{Obliczenia} & \textbf{Pomiar}\\
\hline
$1707,7\Omega$ & $1684\Omega$\\
\hline
\end{tabular}

\subsection{Podsumowanie}
W zadaniu widać było różnice pomiędzy obliczoną rezystancją i jej zmierzoną, rzeczywistą wartością. Różnice te mogą wynikać między innymi z 
:
\begin{itemize}
\item Niezerowego oporu przewodów płytki prototypowej i kabli użytych do podłączenia miernika;
\item Niepewności pomiarowych sprzętu użytego w doświadczeniu;
\item Ograniczonej precyzji wykonania opornika prowadzącej do różnicy między oporem nominalnym a rzeczywistym.
\end{itemize}

\section{Zadanie C}
\subsection*{Cel}
Zadanie C ma na celu zaznajomienie z obsługą zestawu laboratoryjnego NDN DF6911 (konkretnie sekcji DC POWER SUPPLY) i naukę wykonywania pomiarów napięcia i natężenia prądu stałego.


\subsection{Część I}
Odczyt i pomiar napięcia prądu zasilacza.

\begin{center}
\begin{tabular}{|c|c|c|}
\hline
\textbf{U} & \textbf{Pomiar} & \textbf{Odczyt}\\
\hline
1[V] & 1,203V & 1V\\
\hline
3[V] & 3,24V & 3V\\
\hline
4,5[V] & 4,75V & 4,5V\\
\hline
11[V] & 11,23V & 11V\\
\hline
13[V] & 13,22V & 13V\\
\hline
25[V] & 25,31V & 25V\\
\hline
28[V] & 28,27V & 28V\\
\hline
\end{tabular}
\end{center}

\subsection{Część II}
Wyprowadzenie wzoru i zależności opisujących dzielnik napięcia przedstawiony na schemacie poniżej, konstrukcja tego dzielnika oraz próba wyznaczenia $R_{1}$ oraz $R_{2}$, dla których $U_{in}$ = $15V$ i $U_{out}$ = $3,3V$.\\

\begin{figure}[!h]
\centering
\begin{circuitikz}[scale=1.1, font = \scriptsize]

\draw (0,2) to [american voltage source, l_=$U_{in}$] (0,-2)
	  (0,2) -- (1.4, 2) to [R, l=$R_1$] (1.4, 0) to [R, l=$R_2$] (1.4,-2) -- (0,-2) 
	  (1.4,0) -- (2.8,0) -- (2.8,-0.75)
	  (2.8,-1.25) -- (2.8,-2) -- (1.4,-2);
\draw (2.8,-1) node {$U_{out}$};
\end{circuitikz}
\end{figure}
$$U_{in} = R_{1}\cdot I + R_{2}\cdot I$$ 
$$U_{out} = I\cdot R_{2} $$ 
$$I = \frac{U_{out}}{R_{2}}$$ 
$$U_{in} = R_{1} \cdot \frac{U_{out}}{R_{2}} + U_{out}$$ 
$$\frac{R_{1}}{R_{2}} = \frac{U_{in} - U_{out}}{U_{out}}$$
$$\frac{R_{1}}{R_{2}} = \frac{15V - 3,3V}{3,3V} = \frac{11,7V}{3,3V} = 3,54V$$\\

Zatem, znając stosunek $\frac{R_{1}}{R_{2}}$, możemy spróbować uzyskać żądane napięcie używając oporników $R_{1} = 7k\Omega$ i $R_{2} = 2k\Omega$.\\
Pomiar rezystancji dla tak skonstruowanego dzielnika napięcia wynosił $3,365V$.

\subsection{Część III}

Zbudowanie na płytce prototypowej obwodu przedstawionego poniżej, pomiar spadku napięcia na rezystorze $R_{1}$ oraz natężenia prądu w obwodzie.

\begin{figure}[!h]
\centering
\begin{circuitikz}[scale=1.1, font = \scriptsize]
\draw (0,1) to [american voltage source, l_=$V_1$, a^=$5V$] (0,-1)
	  (0,1) to [R, l=$R_1$, a=2100$\Omega$] (3,1) -- (3,-1) -- (0,-1);
\end{circuitikz}
\end{figure}

\begin{flushleft}
Ponieważ opornik $R_{1}$ jest jedynym opornikiem w tym obwodzie, to spadek napięcia na nim wyniesie wartość napięcia źródła, czyli $U_{R1}=5V$. Stąd można obliczyć natężenie w następujący sposób:
\end{flushleft}
$$
I=\frac{U}{R}=\frac{5V}{2100\Omega}=2,381mA
$$
\\
\begin{tabular}{|c|c|}
\hline
\textbf{Natężenie obliczone} & \textbf{Pomiar}\\
\hline
$2,381mA$ & $2,424mA$\\
\hline
\end{tabular}

\subsection{Część IV}
Konstrukcja obwodu przedstawionego na schemacie i sprawdzenie dla niego praw Kirchhoffa (obliczanie spadków napięć na rezystorach i prądów w gałęziach oraz zestawienie ich z pomiarami).

\begin{figure}[!h]
\centering
\begin{circuitikz}[scale=1.1, font = \scriptsize]
\draw (0,1) to [american voltage source, l_=$V_1$, a^=$5V$] (0,-1)
	  (0,1) to [R, l=$R_4$, a=1k$\Omega$] (3,1) to [R, l=$R_2$, a=2k$\Omega$] (5,1) -- (5,-1)
	  (3,-1) to [R, l=$R_3$, a=200$\Omega$] (3,1)
	  (0,-1) to [R, l=$R_1$, a=100$\Omega$] (3,-1) -- (5,-1);
\end{circuitikz}
\end{figure}

%Zadanie C, część IV

%\begin{figure}[!h]
%\centering
%\begin{tikzpicture}[
%    circuit ee IEC,
%    x = 1cm, y = 1cm,
%    every info/.style = {font = \scriptsize},
%    set diode graphic = var diode IEC graphic,
%    set make contact graphic = var make contact IEC graphic]
%
%\draw (0,0) circle [radius = 0.5] 
%	  (0,0.5) to (0,1) --
%	  (0,1) to [resistor={ohm = 1k, info'={$R4$}}] (3,1) --
%	  (3,1) to [resistor={ohm = 200, info'={$R3$}}](3,-1) -- 
%	  (3,-1) to [resistor={ohm = 100, info'={$R1$}}](0,-1) --
%	  (0,-1) to (0,-0.5)
%	  (3,1) to [resistor={ohm = 2k, info'={$R2$}}] (6,1) --
%	  (6,1) to (6,-1) --
%	  (6,-1) to (3,-1);
%\path (-0.5,0.7) node(x) {{V1}}
%	  (-0.5,-0.7) node(x) {{5V}}
%	  (0,-0.28) node(x) {{$\Large -$}}
%	  (0,0.28) node(x) {{$\Large +$}}
%	  ;
%\end{tikzpicture}
%\end{figure}

\begin{flushleft}
Oznaczmy prąd płynący przez R4 jako I1, przez R3 jako I2 i przez R2 jako I3.
Z II prawa Kirchhoffa wiemy, że:
\end{flushleft}

$$
5V - I_{1}\cdot(R_{1}+R_{4}) - R_{3}\cdot(I_{1} - I_{3}) = 0
$$
$$
5V - I_{1}\cdot(R_{1}+R_{4}) - I_{3}\cdot R_{2} = 0\\
$$
$$
I_{1}=11I_{3}
$$
$$
5V=11I_{3}\cdot1100\Omega + 10I_{3}\cdot200\Omega \Rightarrow I_{3}=0,355mA
$$

\begin{flushleft}
Zatem, obliczając wszystkie wartości i zestawiając je z wynikami pomiarów, można dokonać następującego porównania:
\end{flushleft}

\begin{center}
\begin{tabular}{|c|c|c|}
\hline
\textbf{Wielkość fizyczna} & \textbf{Obliczenia} & \textbf{Pomiar}\\
\hline
$I_{1}[mA]$ & $3,905mA$ & $3,962mA$\\
\hline
$I_{2}[mA]$ & $3,55mA$ & $3,587mA$\\
\hline
$I_{3}[mA]$ & $0,355mA$ & $0,362mA$\\
\hline
$U_{R1}[V]$ & $0,391V$ & $0,394V$\\
\hline
$U_{R2}[V]$ & $0,71V$ & $0,712V$\\
\hline
$U_{R3}[V]$ & $0,71V$ & $0,712V$\\
\hline
$U_{R4}[V]$ & $3,905V$ & $3,938V$\\
\hline
\end{tabular}
\end{center}

\begin{flushleft}
Powyższą tabelę można uznać za empiryczny dowód II prawa Kirchhoffa, gdyż drobne różnice między pomiarami i obliczeniami wynikają z niedoskonałości sprzętu pomiarowego i zaokrągleń przy obliczeniach.
\end{flushleft}


\section{Bibliografia}
W trakcie przeprowadzania doświadczeń i pisania sprawozdania zespół korzystał głównie z materiałów ze strony http://mariusznaumowicz.ddns.net/materialy.html oraz z wiedzy własnej.
\bibliographystyle{IEEEtran}

\bibliography{IEEEabrv,refs}

\end{document}

