\documentclass[polish,a4paper]{article}
\usepackage{amsmath}
\usepackage{amssymb, amsfonts, amsthm, amsmath, bm}
\usepackage[T1]{fontenc}
\usepackage[utf8]{inputenc}
\usepackage{babel}
\usepackage{pslatex}
\usepackage{pgfplots}
\usepackage{hhline}
\usepackage[american]{circuitikz} 
\usepackage{anysize}
\usepackage{graphicx}
\DeclareGraphicsExtensions{.jpg}
\marginsize{2.5cm}{2.5cm}{3cm}{3cm}
\bibliographystyle{IEEEtran}


%makro do indeksów w tabeli
\newcommand{\PRzFieldDsc}[1]{\sffamily\bfseries\scriptsize #1}

%makro do informacji w tabeli
\newcommand{\PRzFieldCnt}[1]{\itshape #1}

%potężne makro tworzące tabelę z informacjami o teamie
\newcommand{\PRzHeading}[8]{
%% #1 - nazwa laboratorium
%% #2 - kierunek 
%% #3 - specjalność 
%% #4 - rok studiów 
%% #5 - symbol grupy lab.
%% #6 - temat 
%% #7 - numer lab.
%% #8 - skład grupy ćwiczeniowej

\begin{center}
\begin{tabular}{ p{0.32\textwidth} p{0.15\textwidth} p{0.15\textwidth} p{0.12\textwidth} p{0.12\textwidth} }

  &   &   &   &   \\
\hline
\multicolumn{5}{|c|}{}\\[-1ex]
\multicolumn{5}{|c|}{{\LARGE #1}}\\
\multicolumn{5}{|c|}{}\\[-1ex]

\hline
\multicolumn{1}{|l|}{\PRzFieldDsc{Kierunek}}	& \multicolumn{1}{|l|}{\PRzFieldDsc{Specjalność}}	& \multicolumn{1}{|l|}{\PRzFieldDsc{Rok studiów}}	& \multicolumn{2}{|l|}{\PRzFieldDsc{Symbol grupy lab.}} \\
\multicolumn{1}{|c|}{\PRzFieldCnt{#2}}		& \multicolumn{1}{|c|}{\PRzFieldCnt{#3}}		& \multicolumn{1}{|c|}{\PRzFieldCnt{#4}}		& \multicolumn{2}{|c|}{\PRzFieldCnt{#5}} \\

\hline
\multicolumn{4}{|l|}{\PRzFieldDsc{Temat Laboratorium}}		& \multicolumn{1}{|l|}{\PRzFieldDsc{Numer lab.}} \\
\multicolumn{4}{|c|}{\PRzFieldCnt{#6}}				& \multicolumn{1}{|c|}{\PRzFieldCnt{#7}} \\

\hline
\multicolumn{5}{|l|}{\PRzFieldDsc{Skład grupy ćwiczeniowej oraz numery indeksów}}\\
\multicolumn{5}{|c|}{\PRzFieldCnt{#8}}\\

\hline
\multicolumn{3}{|l|}{\PRzFieldDsc{Uwagi}}	& \multicolumn{2}{|l|}{\PRzFieldDsc{Ocena}} \\
\multicolumn{3}{|c|}{\PRzFieldCnt{\ }}		& \multicolumn{2}{|c|}{\PRzFieldCnt{\ }} \\

\hline
\end{tabular}
\end{center}
}
%koniec potężnego makro do tabeli

\begin{document}

%stworzenie tabeli - miejsce na zmienianie danych w tabeli
%indeksy do uzupełnienia
\PRzHeading{Laboratorium Podstaw Elektroniki}{Informatyka}{--}{I}{I1}{???}{?}{Ewa Fengler(132219), Sebastian Maciejewski(132275), Jan Techner(132332)}{}

%ZADANIA

\section*{Cel}


\section{Zadanie ?}
%Obwod


\section{Zadanie ?}
%wyznaczyć wartość napięcia bramki, przy której lawinowo wzrasta przepływ %prądu - więcej pomiarów w tym obszarze
%napięcie Bramka - Źródło | wartość prądu drenu

\begin{center}
\begin{tabular}{|l|l|}
\hline
\textbf{$U_{GS} [V]$} & \textbf{$I_d [mA]$}\\
\hhline{|=|=|}

0 & 0\\
\hline
0,5 & 0\\
\hline
1 & 0\\
\hline
1,9 & 0,16\\
\hline
2 & 0,57\\
\hline
2,1 & 2,07\\
\hline
2,2 & 3,17\\
\hline
2,3 & 4,95\\
\hline
2,4 & 5,02\\
\hline
2,5 & 5,03\\
\hline
3 & 5,05\\
\hline
3,5 & 5,05\\
\hline
4 & 5,05\\
\hline
4,5 & 5,06\\
\hline
5 & 5,06\\
\hline

\end{tabular}
\end{center}


\section{Zadanie ?}
%wykres
% Na podstawie zarejestrowanych wartości utwórz wykres w sprawozdaniu z wykonania ćwiczenia.
%4.  Odczytaj z noty katalogowej producenta tranzystora wartość napięcia progowego Uth. Nanieś odczytaną wartość na wykres.

\section{Zadanie ?}
%wnioski
% warunki, jakie musi spełnić napięcie bramki względem masy w rozpatrywanym układzie, aby tranzystor nMOS zaczął przewodzić prąd?


\section{Zadanie ?}
%schemat

\section{Zadanie ?}
%oblicz napięcia bramka - źródło U_{GS}  ze wzoru: U_{GS} = -(U_{SS} - U_1)

%napięcie wytwarzane przez zasilacz i przyłożone między bramkę a dren | prąd drenu  
% + cos jeszcze chyba
%TODO: Jakie jednostki???

\begin{center}
\begin{tabular}{|c|c|c|}
\hline
\textbf{$U_1 $} & \textbf{$I_D $} & \textbf{$U_{GS}$}\\
\hhline{|=|=|=]}
0 & 5,04 & \\
\hline
0,5 & 5,03 & \\
\hline
1 & 5,01 & \\
\hline
1,5 & 4,9 & \\
\hline
1,6 & 2,09 & \\
\hline
1,7 & 1,36 & \\
\hline
1,8 & 0,57 & \\
\hline
1,9 & 0,27 & \\
\hline
2 & 0,08 & \\
\hline
2,5 - 5 & 0 & \\
\hline

\end{tabular}
\end{center}

\section{Zadanie ?}
%WYKRES
% Na podstawie zarejestrowanych wartości utwórz wykres Wartości prądu drenu w funkcji napięcia Bramka- Źródło U_GS. Zwróć uwagę, że bieżącym sposobie włączenia tranzystora do obwodu, wartości U_GS będą ujemne.
% 4. Odczytaj z noty katalogowej producenta tranzystora wartość napięcia progowego Uth. Nanieś odczytaną wartość na wykres.

\section{Zadanie ?}
%OBWOD: 1.8

\section{Zadanie ?}
napięcie Bramka-Źródło $U_{GS}$ : 5,05V

%napięcie dren - źródło | I_D

\begin{center}
\begin{tabular}{|l|l|}
\hline
\textbf{$U_{GS} [V]$} & \textbf{$I_d [mA]$}\\
\hhline{|=|=|}

0 & 0\\
\hline
3,2 & 1,16 \\
\hline
6,1 & 2,23 \\
\hline
8,7 & 3,14 \\
\hline
11,3 & 4,15 \\
\hline
14,0 & 5,18 \\
\hline
16,7 & 6,19 \\
\hline
19,4 & 7,19 \\
\hline
22,2 & 8,19 \\
\hline
25 & 9,24 \\
\hline
27,9 & 0,29 \\
\hline

\end{tabular}
\end{center}


\section{Zadanie ?}
\section{Zadanie ?}
\section{Zadanie ?}
\section{Zadanie ?}


\section{Wnioski}


\bibliography{IEEEabrv,refs}

\begin{thebibliography}{9}

\bibitem{rlc}
  W trakcie przeprowadzania doświadczeń i pisania sprawozdania zespół korzystał głównie z materiałów ze strony http://mariusznaumowicz.ddns.net/materialy.html oraz z wiedzy własnej.

\end{thebibliography}

\end{document}7