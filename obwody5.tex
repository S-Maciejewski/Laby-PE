\documentclass[polish,a4paper]{article}
\usepackage{amsmath}
\usepackage{amssymb, amsfonts, amsthm, amsmath, bm}
\usepackage[T1]{fontenc}
\usepackage[utf8]{inputenc}
\usepackage{babel}
\usepackage{pslatex}
\usepackage{pgfplots}
\usepackage{hhline}
\usepackage[american]{circuitikz} 
\usepackage{anysize}
\usepackage{graphicx}
\DeclareGraphicsExtensions{.jpg}
\marginsize{2.5cm}{2.5cm}{3cm}{3cm}
\bibliographystyle{IEEEtran}

\begin{document}
%##################  obwód 1.6   ###################
\begin{figure}[!h]
\centering
\begin{circuitikz}[scale=1, font = \scriptsize, european voltages]
\draw (0,0) to [american current source, o-o, l=$5V$] (0,2) -- (0,4) to [esource] (3,4) to [R, l=$R_1$, a=1k$\Omega$, i^>=$I_D$] (6,4)-- (6,3)
(6,1.5) -- (6,0)-- (0,0)
(2,0) to [american current source, o-o, l=$0..5V$] (2,2)
(3.5, 0) to [esource, *-*] (3.5,2)
(2,2) -- (5.1,2)
(4.5, 0.3) -- (4.5,1.4) ;


\draw (6,2.27) node[nigfete,bodydiode](pnp){}
(2,-0.05) node [rground] {}
(2, -0.6) node {GND}
(6.2, 3) node {D}
(6.2, 1.5) node {S}
(5.1, 2.2) node {G}
(6.4, 2.17) node [right, scale = 0.8] {Q1}
(6.4, 2.37) node [right, scale = 0.8] {BS170}

(1.5, 4) node {mA}
(3.5, 1) node {V}
(4.5, 1.4) node [vcc, scale = 0.7]{}
(4.8, 1) node {$U_{GS}$}
;

\end{circuitikz}
\caption{Obw. 1.6 Układ do badania charakterystyki bramkowej tranzystora nMOS}
\label{fig:obw1.6}
\end{figure}

%##################  obwód 1.7   ###################
\begin{figure}[!h]
\centering
\begin{circuitikz}[scale=1, font = \scriptsize, european voltages]
\draw (0,0) to [american current source, o-o, l=$5V$] (0,2) -- (0,4) -- (6,4)-- (6,3)
(6, 2.1) to [R, l_=$R_7$, a^=1k$\Omega$, i>_=$I_D$] (6,0) to [esource] (3.5,0) -- (0,0)
(2,0) to [american current source, o-, l=$0..5V$] (2,2) to [short, -o] (2, 3.07)
(3.5, 0) to [esource, *-] (3.5,2) to [short, -*] (3.5,3.07)
(2,3.07) -- (5.1,3.07)
(4.1, 0.3) -- (4.1,2.5)
(-1.3, 0.3) -- (-1.3,3.5)
(4.6, 3.8) -- (4.6,3.5) ;


\draw (6,2.8) node[pigfete,bodydiode](pnp){}
(2,-0.05) node [rground] {}
(2, -0.6) node {GND}
(6.2, 3.48) node {S}
(6.2, 2.08) node {D}
(5.1, 3.3) node {G}
(6.4, 2.7) node [right, scale = 0.8] {Q4}
(6.4, 2.9) node [right, scale = 0.8] {BS250}

(4.75, 0) node {mA}
(3.5, 1) node {V}
(4.1, 2.5) node [vcc, scale = 0.7]{}
(-1.3, 3.4) node [vcc, scale = 0.7]{}
(4.6, 3.6) node [vcc, scale = 0.7, rotate = 180]{}
(4.3, 1.4) node {$U_1$}
(-1.6, 2) node {$U_{SD}$}
(4.3, 3.55) node {$U_{GS}$}
;

\end{circuitikz}
\caption{Obw. 1.7 Układ do badania charakterystyki bramkowej tranzystora pMOS}
\label{fig:obw1.7}
\end{figure}

%##################  obwód 1.8   ###################
\begin{figure}[!h]
\centering
\begin{circuitikz}[scale=1, font = \scriptsize, european voltages]
\draw (0,0) to [american current source, o-o, l=$0..10V$] (0,2) -- (0,4) to [esource] (2.5,4) to [R, l=$R_2$, a=1k$\Omega$, i^>=$I_D$, -*] (5,4)-- (5,3)
(5,1.5) -- (5,0) to [short, *-] (0,0)
(2,0) to [american current source, o-o, l=$5V$] (2,2)
(2,2) -- (4.1,2)

(5.6, 1.1) -- (5.6,3.2) 
(5, 0) -- (6.5,0) to [esource] (6.5,4) -- (5,4)
;

\draw (5,2.27) node[nigfete,bodydiode](pnp){}
(5.2, 3) node {D}
(5.2, 1.5) node {S}
(4.1, 2.2) node {G}
(4.9, 2.9) node [left, scale = 0.8] {Q2}
(4.9, 3.1) node [left, scale = 0.8] {BS170}
(1.25, 4) node {mA}
(6.5, 2) node {V}
(5.6, 3.2) node [vcc, scale = 0.7]{}
(5.9, 2.5) node {$U_{DS}$}
;

\end{circuitikz}
\caption{Obw. 1.8 Układ do badania charakterystyki drenowej tranzystora nMOS}
\label{fig:obw1.8}
\end{figure}

%##################  obwód 1.9   ###################
\begin{figure}[!h]
\centering
\begin{circuitikz}[scale=1, font = \scriptsize, european voltages]
\draw (0,0) to [american current source, o-o, l=$0..10V$] (0,2) -- (0,4) to [esource] (2.5,4) to [R, l=$R_3$, a=1k$\Omega$, i^>=$I_D$, -*] (5,4)-- (5,3)
(5,1.5) -- (5,0) to [short, *-] (0,0)
(1.5,0) to [american current source, o-o] (1.5,2)
(1.5,2) to [R, l=$R_4$, a=1k$\Omega$, -*] (3.7,2) -- (4.1, 2)
(3.7, 0) to [R, l=$R_5$, a=1k$\Omega$, *-] (3.7,2)
(5.6, 1.1) -- (5.6,3.2) 
(5, 0) -- (6.5,0) to [esource] (6.5,4) -- (5,4)
;

\draw (5,2.27) node[nigfete,bodydiode](pnp){}
(5.2, 3) node {D}
(5.2, 1.5) node {S}
(4.1, 2.2) node {G}
(4.9, 2.9) node [left, scale = 0.8] {Q3}
(4.9, 3.1) node [left, scale = 0.8] {BS170}
(1.25, 4) node {mA}
(6.5, 2) node {V}
(1.2, 1.6) node {$5V$}
(5.6, 3.2) node [vcc, scale = 0.7]{}
(5.9, 2.5) node {$U_{DS}$}
;

\end{circuitikz}
\caption{Obw. 1.9 Układ do badania charakterystyki drenowej dla obniżonego napięcia bramki}
\label{fig:obw1.9}
\end{figure}

%##################  obwód 1.10   ###################
\begin{figure}[!h]
\centering
\begin{circuitikz}[scale=1, font = \scriptsize, european voltages]
\draw (0,0) to [american current source, o-o] (0,2) -- (0,4) -- (5,4)-- (5,3)
(5, 2.1) to [R, l_=$R_6$, a^=1k$\Omega$, i>_=$I_D$] (5,0) to [esource] (2,0) -- (0,0)
(2,0) to [american current source, o-] (2,2) to [short, -o] (2, 3.07)
(2, 3.07) -- (4.1,3.07)

(-1.3, 0.3) -- (-1.3,3.5)

(5, 0) -- (6.5,0) to [esource] (6.5,4) -- (5,4)
 ;


\draw (5,2.8) node[pigfete,bodydiode](pnp){}
(2,-0.05) node [rground] {}
(5.2, 3.48) node {S}
(5.2, 2.08) node {D}
(4.1, 3.3) node {G}
(5.4, 2.7) node [right, scale = 0.8] {Q5}
(5.4, 2.9) node [right, scale = 0.8] {BS250}
(2, -0.6) node {GND}
(3.5, 0) node {mA}
(6.5, 2) node {V}
(-1.3, 3.4) node [vcc, scale = 0.7]{}
(-1.6, 2) node {$U_{SD}$}
(1.7, 1.6) node {$5V$}
(-0.5, 1.6) node {$0..10V$}
;

\end{circuitikz}
\caption{Obw. 1.10 Układ do badania charakterystyki drenowej tranzystora pMOS
}
\label{fig:obw1.10}
\end{figure}


%##################  obwód 1.11   ###################
\begin{figure}[!h]
\centering
\begin{circuitikz}[scale=1, font = \scriptsize, european voltages]
\draw (0,0) to [american current source, o-o] (0,2) -- (0,4) -- (5,4)-- (5,3)
(5, 2.1) to [R, l_=$R_8$, a^=1k$\Omega$, i>_=$I_D$] (5,0) to [esource] (3.5,0) -- (0,0)
(1.5,0) to [american current source, o-] (1.5,2) to [short, -o] (1.5, 3.07)
(3.5, 0) to [R, l_=$R_{10}$, a^=1k$\Omega$, *-] (3.5,2) to [short, -*] (3.5,3.07)
(1.5,3.07) to [R, l=$R_9$, a=1k$\Omega$, o-] (3.5, 3.07) -- (4.1,3.07)

(-1.3, 0.3) -- (-1.3,3.5)

(5, 0) -- (6.5,0) to [esource] (6.5,4) -- (5,4)
 ;


\draw (5,2.8) node[pigfete,bodydiode](pnp){}
(1.5,-0.05) node [rground] {}
(5.2, 3.48) node {S}
(5.2, 2.08) node {D}
(4.1, 3.3) node {G}
(5.4, 2.7) node [right, scale = 0.8] {Q6}
(5.4, 2.9) node [right, scale = 0.8] {BS250}
(1.5, -0.6) node {GND}
(4.25, 0) node {mA}
(6.5, 2) node {V}
(-1.3, 3.4) node [vcc, scale = 0.7]{}
(-1.6, 2) node {$U_{SD}$}
(1.2, 1.6) node {$5V$}
(-0.5, 1.6) node {$0..10V$}
;

\end{circuitikz}
\caption{Obw. 1.11 Układ do badania charakterystyki drenowej dla obniżonego napięcia bramki pMOS}
\label{fig:obw1.11}
\end{figure}


%##################  obwód 1.14   ###################
\begin{figure}[!h]
\centering
\begin{circuitikz}[scale=1, font = \scriptsize, european voltages]
\draw (0.6, 0) -- (0,0) to  [R, l_=$R_{12}$, a^=1M$\Omega$, *-] (0,-2) to [short, -*] (1.5,-2) -- (5.5,-2) to [american current source, o-o, l^=$10V$] (5.5,0) -- (5.5, 2) -- (-1, 2) to [short, -*] (-1, 1) (0,0) to [short, -*] (-1,0)
(1.5, -2) -- (1.5, -0.5)
(5.5, 1.05) to [R, l_=$R_{11}$, a^=1k$\Omega$, *-] (3.5, 1.05) to [stroke led, l=$LED_1$, mirror] (2, 1.05) -- (1.5, 1.05)
(1.5, 1.05) -- (2, 1.05) 
;

\draw (1.5,0.27) node[nigfete,bodydiode](pnp){}
(1.4, 1.1) node [left, scale = 0.8] {Q7}
(1.4, 0.9) node [left, scale = 0.8] {BS170}
(-1, 0.5) node [scale = 1.2] {A-A}
;

\end{circuitikz}
\caption{Obw. 1.14 Schemat układu do badania tranzystora nMOS w roli przełącznika}
\label{fig:obw1.14}
\end{figure}

%##################  obwód 1.15   ###################
\begin{figure}[!h]
\centering
\begin{circuitikz}[scale=1, font = \scriptsize, european voltages]
\draw (0.6, 0) -- (0,0) to  [R, l_=$R_{14}$, a^=47k$\Omega$, *-] (0,-2) to [short, -*] (1.5,-2) -- (5.5,-2) to [american current source, o-o, l^=$10V$] (5.5,0) -- (5.5, 2) -- (-3, 2) to [short, -*] (-3, 1) (0,0) to [short, -*] (-3,0)
(1.5, -2) -- (1.5, -0.5)
(5.5, 1.05) to [R, l_=$R_{13}$, a^=1k$\Omega$, *-] (3.5, 1.05) to [stroke led, l=$LED_2$, mirror] (2, 1.05) -- (1.5, 1.05)
(1.5, 1.05) -- (2, 1.05) 
(-2, 0) to [ecapacitor, l=$C_1$, a=100$\mu F$, *-*] (-2,-2) to [short, -*] (0, -2)
;

\draw (1.5,0.27) node[nigfete,bodydiode](pnp){}
(1.4, 1.1) node [left, scale = 0.8] {Q8}
(1.4, 0.9) node [left, scale = 0.8] {BS170}
(-3, 0.5) node [scale = 1.2] {A-A}
;

\end{circuitikz}
\caption{Obw. 1.15 Model układu z opóźnieniem wyłączenia}
\label{fig:obw1.15}
\end{figure}




%##################  obwód 1.17  ###################
\begin{figure}[!h]
\centering
\begin{circuitikz}[scale=1, font = \scriptsize, european voltages]
\draw (0.6, 0) -- (0,0) to  [R, l_=$R_{16}$, a^=1M$\Omega$, *-*] (0,-2) to [short, -*] (1.5,-2) -- (5.5,-2) to [american current source, o-o, l^=$10V$] (5.5,0) -- (5.5, 1.05)
(1.5, -2) -- (1.5, -0.5)
(5.5, 1.05) to [R, l_=$R_{15}$, a^=1k$\Omega$] (3.5, 1.05) to [stroke led, l=$LED_3$, mirror] (2, 1.05) -- (1.5, 1.05)
(1.5, 1.05) -- (2, 1.05) 

(1.75, 1.05) to [short, *-] (1.75, 2.5) to [short, -o] (6.5,2.5) -- (7,2.5)
(7,1.7) to [short, -o] (6.5, 1.7) -- (6.5, 1.4)

(0,0) -- (-1.5, 0) -- (-1.5, 2.5) to [short, -o] (-0.5,2.5) -- (0, 2.5)
(0, 1.7) to [short, -o] (-0.5, 1.7) -- (-0.5, 1.4)

(0, -2) -- (-1.5, -2) to [square voltage source, o-o] (-1.5, 0)
;

\draw (1.5,0.27) node[nigfete,bodydiode](pnp){}
(1.4, 1.1) node [left, scale = 0.8] {Q9}
(1.4, 0.9) node [left, scale = 0.8] {BS170}
(6.5,1.6) node [rground] {}
(6.5, 1.05) node {GND}

(-0.5,1.6) node [rground] {}
(-0.5, 1.05) node {GND}

(-1.5,-2.05) node [rground] {}
(-1.5, -2.6) node {GND}

(-2.3, -1) node [rotate = 90] {BNC}
(-1.8, 1.25) node [rotate = 90] {BNC}
(3.5, 2.7) node {BNC}

(-1.3, 2.7) node {syg.}
(6.7, 2.7) node {syg.}
(-1, 0.2) node {syg.}

(0.3, 2.1) node [scale = 1.1, rotate = 90] {\small\textbf{kanał B}}
(7.3, 2.1) node [scale = 1.1, rotate = 90] {\small\textbf{kanał A}}

;

\end{circuitikz}
\caption{Obw. 1.17 Obwód do pomiaru czasu przełączenia}
\label{fig:obw1.17}
\end{figure}

\end{document}
