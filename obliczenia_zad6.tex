\documentclass[12pt]{article}
\usepackage{amsmath}
\usepackage{amssymb, amsfonts, amsthm, amsmath, bm}
\usepackage[polish]{babel}
\usepackage[cp1250]{inputenc}
\usepackage[T1]{fontenc}
\begin{document}

%$5V = IR_4 + I_2R_2 + JR_3$\\
%$5V = IR_4 + I_1R_1 + IR_3$\\
%$I_1 + I_2 = I$

%\begin{equation*}
%5V = IR_4 + I_2R_2 + JR_3\\
%5V = IR_4 + I_1R_1 + IR_3\\
%I_1 + I_2 = I\\
%\end{equation*}


$ \begin{cases} 
\phantom{I_1 + }
	   5V& = IR_4 + I_2R_2 + JR_3\\
\phantom{I_1 + }
	   5V& = IR_4 + I_1R_1 + IR_3\\
I_1 + I_2& = I
\end{cases} $

\bigbreak
\bigbreak

$ \begin{cases}
\phantom{I_1 + .}
		0 &= I_2R_2 - I_1R_1\\
I_1 + I_2 &= I\\
\phantom{I_1 + }
	   5V &= IR_4 + I_1R_1 + IR_3
\end{cases} $

\bigbreak
\bigbreak

$ \begin{cases}
\phantom{I. }
   I_1R_1 &= (I - I_1)R_2\\
I_1 + I_2 &= I\\
\phantom{I_1 + }
	   5V &= IR_4 + (I - I_1)R_2 + IR_3
\end{cases} $

\bigbreak
\bigbreak


$\bm{R_Z} = R_4 + \frac{R_2R_1}{R_1 + R_2} + R_3 = 610 = \frac{220*360}{220 = 360} = \bm{747\Omega}$
%nie za mało dokładnie?
\bigbreak

$R = \frac{U}{I}\quad\Rightarrow\quad \bm{I} = \frac{U}{R_Z} = \frac{5V}{747\Omega} = 6,69mA = \bm{0,00669A}$
\bigbreak
\bigbreak

$5V = IR_4 + I_2R_2 + IR_3$
\bigbreak

$5V = IR_4 + I_1R_1 + IR_3$
\bigbreak
\bigbreak

$\frac{5V - I(R_4 + R_3)}{R_2} = \bm{I_2 = 4,177mA}$
\bigbreak

$\frac{5V - I(R_4 + R_3)}{R_1} = \bm{I_1 = 2,55mA}$


\end{document}


