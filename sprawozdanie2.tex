\documentclass[polish,a4paper]{article}
\usepackage{amsmath}
\usepackage{amssymb, amsfonts, amsthm, amsmath, bm}
\usepackage[T1]{fontenc}
\usepackage[utf8]{inputenc}
\usepackage{babel}
\usepackage{pslatex}
\usepackage{pgfplots}
\usepackage[american]{circuitikz} 
\usepackage{anysize}
\marginsize{2.5cm}{2.5cm}{3cm}{3cm}

%makro do indeksów w tabeli
\newcommand{\PRzFieldDsc}[1]{\sffamily\bfseries\scriptsize #1}

%makro do informacji w tabeli
\newcommand{\PRzFieldCnt}[1]{\itshape #1}

%potężne makro tworzące tabelę z informacjami o teamie
\newcommand{\PRzHeading}[8]{
%% #1 - nazwa laboratorium
%% #2 - kierunek 
%% #3 - specjalność 
%% #4 - rok studiów 
%% #5 - symbol grupy lab.
%% #6 - temat 
%% #7 - numer lab.
%% #8 - skład grupy ćwiczeniowej

\begin{center}
\begin{tabular}{ p{0.32\textwidth} p{0.15\textwidth} p{0.15\textwidth} p{0.12\textwidth} p{0.12\textwidth} }

  &   &   &   &   \\
\hline
\multicolumn{5}{|c|}{}\\[-1ex]
\multicolumn{5}{|c|}{{\LARGE #1}}\\
\multicolumn{5}{|c|}{}\\[-1ex]

\hline
\multicolumn{1}{|l|}{\PRzFieldDsc{Kierunek}}	& \multicolumn{1}{|l|}{\PRzFieldDsc{Specjalność}}	& \multicolumn{1}{|l|}{\PRzFieldDsc{Rok studiów}}	& \multicolumn{2}{|l|}{\PRzFieldDsc{Symbol grupy lab.}} \\
\multicolumn{1}{|c|}{\PRzFieldCnt{#2}}		& \multicolumn{1}{|c|}{\PRzFieldCnt{#3}}		& \multicolumn{1}{|c|}{\PRzFieldCnt{#4}}		& \multicolumn{2}{|c|}{\PRzFieldCnt{#5}} \\

\hline
\multicolumn{4}{|l|}{\PRzFieldDsc{Temat Laboratorium}}		& \multicolumn{1}{|l|}{\PRzFieldDsc{Numer lab.}} \\
\multicolumn{4}{|c|}{\PRzFieldCnt{#6}}				& \multicolumn{1}{|c|}{\PRzFieldCnt{#7}} \\

\hline
\multicolumn{5}{|l|}{\PRzFieldDsc{Skład grupy ćwiczeniowej oraz numery indeksów}}\\
\multicolumn{5}{|c|}{\PRzFieldCnt{#8}}\\

\hline
\multicolumn{3}{|l|}{\PRzFieldDsc{Uwagi}}	& \multicolumn{2}{|l|}{\PRzFieldDsc{Ocena}} \\
\multicolumn{3}{|c|}{\PRzFieldCnt{\ }}		& \multicolumn{2}{|c|}{\PRzFieldCnt{\ }} \\

\hline
\end{tabular}
\end{center}
}
%koniec potężnego makro do tabeli

\begin{document}

%stworzenie tabeli - miejsce na zmienianie danych w tabeli
%indeksy do uzupełnienia
\PRzHeading{Laboratorium Podstaw Elektroniki}{Informatyka}{--}{I}{I1}{Twierdzenie Thevenina}{2}{Ewa Fengler(132219), Sebastian Maciejewski(132275), Jan Techner(132332)}{}

%ZADANIA

\section*{Cel}
Celem przeprowadzanych doświadczeń jest zaznajomienie się z twierdzeniem Thevenina oraz jego zastosowaniem do pomiaru prądów w gałęziach.

\section{Zadanie 1.}
Rozpatrywany obwód wraz z wybranymi wartościami elementów.

\begin{figure}[!h]
\centering
\begin{circuitikz}[scale=1.1, font = \scriptsize]
\draw (0,2) to [american voltage source, l_=$V_1$, a^=$5V$] (0,-2)
	  (0,2) to [R, l=$R_4$, a=100$\Omega$] (2,2) to [R, l_=$R_2$, a^=220$\Omega$] (2,0) to [R, l_=$R_3$, a^=510$\Omega$] (2,-2) -- (0,-2)
	  (2,2) -- (4,2) to [R, l_=$R_1$, a^=360$\Omega$] (4,0) -- (2,0);
\end{circuitikz}
\end{figure}

\section{Zadanie 2.}
Wartości rezystorów użytych do zbudowania obwodu.

\begin{center}
\begin{tabular}{|c|c|c|c|c|}
\hline
\textbf{Lp.} & \textbf{R} & \textbf{Kod paskowy(KP)} & \textbf{Wartość odczytana z KP} & \textbf{Wartość zmierzona}\\
\hline
1. & $R_1$ & pomarańczowy, niebieski, brązowy, złoty & $360\Omega\pm5\%$ & $354,9\Omega$\\
\hline
2. & $R_2$ & czerwony, czerwony, czarny, złoty & $220\Omega\pm5\%$ & $218\Omega$\\
\hline
3. & $R_3$ & zielony, brązowy, brązowy, złoty & $510\Omega\pm5\%$ & $499,9\Omega$\\
\hline
4. & $R_4$ & brązowy, czarny, brązowy, złoty & $100\Omega\pm5\%$ & $97,5\Omega$\\
\hline
\end{tabular}
\end{center}

\newpage

\section{Zadanie 3.}
Wyniki pomiarów dla twierdzenia Thevenina.

\begin{center}
\begin{tabular}{|c|c|c|}
\hline
\textbf{Lp.} & \textbf{$U_{th}$} & \textbf{$R_{th}$}\\
\hline
1. & $1,35V$ & $159,99\Omega$\\
\hline
12. & $1,88V$ & $222,91\Omega$\\
\hline
\end{tabular}
\end{center}

\section{Zadanie 4.}
Obliczenie prądów dla badanego obwodu w gałęzi z rezystorem $R_x$ w oparciu o twierdzenie Thevenina.

\begin{figure}[!h]
\centering
\begin{circuitikz}[scale=1.1, font = \scriptsize]
\draw (0,1) to [american voltage source, l_=$U_{th}$] (0,-1)
	  (0,1) to [R, l=$R_{th}$] (2,1) to [R, l=$R_x$] (2,-1) -- (0,-1);
\end{circuitikz}
\end{figure}

$$
I_{R1} = \frac{U_{th1}}{R_{th1}+R_1} = \frac{1,35V}{159,99\Omega + 354,9\Omega} = 2,62mA
$$
$$
I_{R2} = \frac{U_{th2}}{R_{th2}+R_2} = \frac{1,88V}{222,91\Omega + 218\Omega} = 4,26mA
$$

\section{Zadanie 5.}
Zestawienie wyników z poprzednich zadań.
\begin{center}
\begin{tabular}{|c|c|c|c|}
\hline
\textbf{Lp.} & \textbf{$U_{th}$} & \textbf{$R_{th}$} & \textbf{$I_{Rx}$}\\
\hline
1. & $1,35V$ & $159,99\Omega$ & $2,62mA$\\
\hline
2. & $1,88V$ & $222,91\Omega$ & $4,26mA$\\
\hline
\end{tabular}
\end{center}

\section{Zadanie 6.}
Analityczne obliczenie wartości szukanych prądów. 
\newline
\newline
Korzystając z I i II prawa Kirchoffa otrzymujemy dla danego obwodu następujący układ równań : 
\bigbreak
\bigbreak
$ \begin{cases} 
\phantom{I_1 + }
	   V_1& = IR_4 + I_2R_2 + JR_3\\
\phantom{I_1 + }
	   V_1& = IR_4 + I_1R_1 + IR_3\\
I_1 + I_2& = I
\end{cases} $

\bigbreak
\bigbreak

$ \begin{cases}
\phantom{I_1 + .}
		0 &= I_2R_2 - I_1R_1\\
I_1 + I_2 &= I\\
\phantom{I_1 + }
	   5V &= IR_4 + I_1R_1 + IR_3
\end{cases} $

\bigbreak
\bigbreak

$ \begin{cases}
\phantom{I. }
   I_1R_1 &= (I - I_1)R_2\\
I_1 + I_2 &= I\\
\phantom{I_1 + }
	   5V &= IR_4 + (I - I_1)R_2 + IR_3
\end{cases} $

\bigbreak
\bigbreak


$\bm{R_Z} = R_4 + \frac{R_2R_1}{R_1 + R_2} + R_3 = 610 = \frac{220*360}{220 = 360} = \bm{747\Omega}$
%nie za mało dokładnie?
\bigbreak

$R = \frac{U}{I}\quad\Rightarrow\quad \bm{I} = \frac{U}{R_Z} = \frac{5V}{747\Omega} = 6,69mA = \bm{0,00669A}$
\bigbreak
\bigbreak

$5V = IR_4 + I_2R_2 + IR_3$
\bigbreak

$5V = IR_4 + I_1R_1 + IR_3$
\bigbreak
\bigbreak

$\frac{5V - I(R_4 + R_3)}{R_2} = \bm{I_2 = 4,177mA}$
\bigbreak

$\frac{5V - I(R_4 + R_3)}{R_1} = \bm{I_1 = 2,55mA}$

\section{Zadanie 7.}
Zestawienie danych otrzymanych w wyniku obliczeń z danymi pomiarowymi.

\begin{center}
\begin{tabular}{|c|c|c|}
\hline
\textbf{Lp.} & \textbf{$I_{Rx}$(z tw. Thevenina)} & \textbf{$I_{Rx}$(z obliczeń)}\\
\hline
1. & $2,62mA$ & $2,55mA$\\
\hline %TU WPISAĆ WYNIKI OBLICZEŃ
2. & $4,26mA$ & $4,177mA$\\
\hline
\end{tabular}
\end{center}

\section{Wnioski}

%TU WPISAĆ WNIOSKI

\section*{Bibliografia}
W trakcie przeprowadzania doświadczeń i pisania sprawozdania zespół korzystał głównie z materiałów ze strony http://mariusznaumowicz.ddns.net/materialy.html oraz z wiedzy własnej.
\bibliographystyle{IEEEtran}

\bibliography{IEEEabrv,refs}

\end{document}

