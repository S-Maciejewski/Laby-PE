\documentclass[polish,a4paper]{article}
\usepackage{amsmath}
\usepackage{amssymb, amsfonts, amsthm, amsmath, bm}
\usepackage[T1]{fontenc}
\usepackage[utf8]{inputenc}
\usepackage{babel}
\usepackage{pslatex}
\usepackage{pgfplots}
\usepackage{hhline}
\usepackage[american]{circuitikz} 
\usepackage{anysize}
\marginsize{2.5cm}{2.5cm}{3cm}{3cm}
\bibliographystyle{IEEEtran}

%makro do indeksów w tabeli
\newcommand{\PRzFieldDsc}[1]{\sffamily\bfseries\scriptsize #1}

%makro do informacji w tabeli
\newcommand{\PRzFieldCnt}[1]{\itshape #1}

%potężne makro tworzące tabelę z informacjami o teamie
\newcommand{\PRzHeading}[8]{
%% #1 - nazwa laboratorium
%% #2 - kierunek 
%% #3 - specjalność 
%% #4 - rok studiów 
%% #5 - symbol grupy lab.
%% #6 - temat 
%% #7 - numer lab.
%% #8 - skład grupy ćwiczeniowej

\begin{center}
\begin{tabular}{ p{0.32\textwidth} p{0.15\textwidth} p{0.15\textwidth} p{0.12\textwidth} p{0.12\textwidth} }

  &   &   &   &   \\
\hline
\multicolumn{5}{|c|}{}\\[-1ex]
\multicolumn{5}{|c|}{{\LARGE #1}}\\
\multicolumn{5}{|c|}{}\\[-1ex]

\hline
\multicolumn{1}{|l|}{\PRzFieldDsc{Kierunek}}	& \multicolumn{1}{|l|}{\PRzFieldDsc{Specjalność}}	& \multicolumn{1}{|l|}{\PRzFieldDsc{Rok studiów}}	& \multicolumn{2}{|l|}{\PRzFieldDsc{Symbol grupy lab.}} \\
\multicolumn{1}{|c|}{\PRzFieldCnt{#2}}		& \multicolumn{1}{|c|}{\PRzFieldCnt{#3}}		& \multicolumn{1}{|c|}{\PRzFieldCnt{#4}}		& \multicolumn{2}{|c|}{\PRzFieldCnt{#5}} \\

\hline
\multicolumn{4}{|l|}{\PRzFieldDsc{Temat Laboratorium}}		& \multicolumn{1}{|l|}{\PRzFieldDsc{Numer lab.}} \\
\multicolumn{4}{|c|}{\PRzFieldCnt{#6}}				& \multicolumn{1}{|c|}{\PRzFieldCnt{#7}} \\

\hline
\multicolumn{5}{|l|}{\PRzFieldDsc{Skład grupy ćwiczeniowej oraz numery indeksów}}\\
\multicolumn{5}{|c|}{\PRzFieldCnt{#8}}\\

\hline
\multicolumn{3}{|l|}{\PRzFieldDsc{Uwagi}}	& \multicolumn{2}{|l|}{\PRzFieldDsc{Ocena}} \\
\multicolumn{3}{|c|}{\PRzFieldCnt{\ }}		& \multicolumn{2}{|c|}{\PRzFieldCnt{\ }} \\

\hline
\end{tabular}
\end{center}
}
%koniec potężnego makro do tabeli

\begin{document}

%stworzenie tabeli - miejsce na zmienianie danych w tabeli
%indeksy do uzupełnienia
\PRzHeading{Laboratorium Podstaw Elektroniki}{Informatyka}{--}{I}{I1}{Rezonans w obwodach RLC}{3}{Ewa Fengler(132219), Sebastian Maciejewski(132275), Jan Techner(132332)}{}

%ZADANIA

\section*{Cel}
-----------------------------------------

\section{Zadanie 1.}
Rozpatrywany obwód wraz z wybranymi wartościami elementów.

\begin{figure}[!h]
\centering
\begin{circuitikz}[scale=1.1, font = \scriptsize]
\draw (-0.3,1) to [sinusoidal voltage source, l^=$V_{pp}$, a_=$8V$, o-o] (-0.3,-1)
	  (-0.3,1) -- (1,1)
	  (-0.3,-1) -- (1,-1)
	  (2,1) to [short, -*] (2.5,1) -- (2.5, 2.5) to [voltmeter, -*] (4, 2.5) to [voltmeter] (5.5, 2.5) to [short, -*] (5.5, 1) to [short, -*] (6,1) to [R, l=$R_1$, a=1k$\Omega$, *-*] (6,-1) -- (2,-1) 
	  (4,2.5) to [short, -*] (4,1) 
	  (2.5,1) to [C, l=$C_1$, a=13.3nF] (4,1) to [L, l=$L_1$, a=66mH] (5.5,1)
	  (6,1) -- (7.5,1) 
	  (6,-1) -- (7.5,-1)
	  (1.5,-2.3) to [short, -o] (3, -2.3)
	  (1.5,-3) to [short, -o] (3, -3)
	  (5,-2.3) to [short, -o] (6.5, -2.3)
	  (5,-3) to [short, -o] (6.5, -3);
\draw [line width = 2, blue] (1,1) -- (1,-1) -- (2,-1) -- (2, 1)
	  (1.5, -1) -- (1.5, -3)
	  (7.5,1) -- (7.5,-1.7) -- (5,-1.7) --(5,-3)  ;  
\draw [line width = 1, dashed, gray] (2.2,1.9) -- (6.7,1.9) -- (6.7,-1.2) -- (2.2, -1.2) -- (2.2,1.9); 
\draw (0.5,1.1) node {czerw.}
	  (0.5,-0.9) node {biały}
	  (7.2,1.1) node {czerw.}
	  (7.2,-0.9) node {biały}
	  
	  (2.6, -2.2) node {czerw.}
	  (2.6, -2.9) node {biały}
	  (6.1, -2.2) node {czerw.}
	  (6.1, -2.9) node {biały}
	  (1.5, -0.8) node {BNC}
	  (5.5,-1.5) node {BNC}
      (1.3, -2) node[rotate=90] {BNC}
      (3.3, -2.65) node[rotate=90] {\small\textbf{kanał X}}
      (6.8, -2.65) node[rotate=90] {\small\textbf{kanał Y}}
      (-1.1, 0.4) node {sinus}
      (-1.45, -0.4) node {f = 1...15kHz} 
	  ;
\end{circuitikz}
\caption{Badany obwód}
\label{fig:badobw}
\end{figure}
$\newline$
Wartości elementów obwodu : $V_{pp} = 8V$, $R_1 = $1k$\Omega$,  $C_1 = 13.3$nF,  $L_1 = 66$mH


\section{Zadanie 2.}
-----------------------------------
%Wartości rezystorów użytych do zbudowania obwodu przedstawionego na rysunku \ref{fig:badobw}.
%
%\begin{center}
%\begin{tabular}{|c||c|c|c|c|}
%\hline
%\textbf{Lp.} & \textbf{R} & \textbf{Kod paskowy(KP)} & \textbf{Wartość odczytana z KP} & \textbf{Wartość zmierzona}\\
%\hhline{|=#=|=|=|=|}
%1. & $R_1$ & pomarańczowy, niebieski, brązowy, złoty & $360\Omega\pm5\%$ & $354,9\Omega$\\
%\hline
%2. & $R_2$ & czerwony, czerwony, czarny, złoty & $220\Omega\pm5\%$ & $218\Omega$\\
%\hline
%3. & $R_3$ & zielony, brązowy, brązowy, złoty & $510\Omega\pm5\%$ & $499,9\Omega$\\
%\hline
%4. & $R_4$ & brązowy, czarny, brązowy, złoty & $100\Omega\pm5\%$ & $97,5\Omega$\\
%\hline
%\end{tabular}
%\end{center}
%\newpage

\section{Zadanie 3.}
------------------------------------
%Wyniki pomiarów dla twierdzenia Thevenina, gdzie: 
%\begin{center}
%\textbf{$U_{th}$} - napięcie panujące od strony zacisków AB \\
%\textbf{$R_{th}$} - rezystencja zastępcza widziana od strony zacisków AB \\
%\end{center}
%
%\begin{center}
%\begin{tabular}{|c||c|c|}
%\hline
%\textbf{Lp.} & \boldsymbol{$U_{th}$} & \boldsymbol{$R_{th}$}\\
%\hhline{|=#=|=|}
%1. & $1,35V$ & $159,99\Omega$\\
%\hline
%2. & $1,88V$ & $222,91\Omega$\\
%\hline
%\end{tabular}
%\end{center}

\section{Zadanie 4.}
-------------------------------------
%Obliczenie prądów dla badanego obwodu w gałęzi z rezystorem $R_x$ w oparciu o twierdzenie Thevenina \cite{thevenin}.
%
%\begin{figure}[!h]
%\centering
%\begin{circuitikz}[scale=1.1, font = \scriptsize]
%\draw (0,1) to [american voltage source, l_=$U_{th}$] (0,-1)
%	  (0,1) to [R, l=$R_{th}$] (2,1) to [R, l=$R_x$] (2,-1) -- (0,-1);
%\end{circuitikz}
%\caption{Schematyczny obwód}
%\label{fig:schemobw}
%\end{figure}
%
%$$
%I_{R1} = \frac{U_{th1}}{R_{th1}+R_1} = \frac{1,35V}{159,99\Omega + 354,9\Omega} = 2,62mA
%$$
%$$
%I_{R2} = \frac{U_{th2}}{R_{th2}+R_2} = \frac{1,88V}{222,91\Omega + 218\Omega} = 4,26mA
%$$

\section{Zadanie 5.}
------------------------------------
%Zestawienie wyników z poprzednich zadań.
%
%\begin{center}
%\begin{tabular}{|c||c|c|c|}
%\hline
%\textbf{Lp.} & \boldsymbol{$U_{th}$} & \boldsymbol{$R_{th}$} & \boldsymbol{$I_{Rx}$}\\
%%\hhline{|=#=|=|=|}
%%1. & $1,35V$ & $159,99\Omega$ & $2,62mA$\\
%%\hline
%%2. & $1,88V$ & $222,91\Omega$ & $4,26mA$\\
%%\hline
%%\end{tabular}
%%\end{center}
%%\newpage
\section{Zadanie 6.}
----------------------------------
%%Analityczne obliczenie wartości szukanych prądów. 
%%\newline
%%\begin{figure}[!h]
%%\centering
%%\begin{circuitikz}[scale=1.1, font = \scriptsize]
%%\draw (0,2) to [american voltage source, l_=$V_1$, i=$I$, a^=$5V$] (0,-2)
%%	  (0,2) to [R, l=$R_4$, i=$I$, a=100$\Omega$] (2,2) to [R, l_=$R_2$, i=$I_2$, a^=220$\Omega$] (2,0) to [R, l_=$R_3$, i=$I$, a^=510$\Omega$] (2,-2) -- (0,-2)
%%	  (2,2) -- (4,2) to [R, l_=$R_1$, i=$I_1$, a^=360$\Omega$] (4,0) -- (2,0);
%%\end{circuitikz}
%%\caption{Badany obwód}
%\label{fig:obw2}
%\end{figure}
%\newline
%\newline
%Korzystając z II prawa Kirchoffa \cite{thevenin} otrzymujemy dla obwodu przedstawionego na rysunku \ref{fig:obw2} następujące równania : 
%\bigbreak
%$V_1 = IR_4 + I_2R_2 + IR_3$\\
%
%$V_1 = IR_4 + I_1R_1 + IR_3$
%\newline
%\newline
%Następnie korzystając ze wzoru na rezystancję zastępczą obwodu oraz zależności rezystancji od napięcia źródła obwodu i natężenia prądu w nim płynącego \cite{thevenin} otrzymujemy :
%\bigbreak
%$R_Z = R_4 + \frac{R_{2} \cdot R_{1}}{R_1 + R_2} + R_3 = 610\Omega + \frac{220\Omega\cdot360\Omega}{220\Omega + 360\Omega} \approx \bm{747\Omega}$
%
%\bigbreak
%
%$R_Z = \frac{V_1}{I}\quad\Rightarrow\quad \bm{I} = \frac{V_1}{R_Z} = \frac{5V}{747\Omega} \approx 6,69mA = \bm{0,00669A}$
%\newline
%\newline
%Po przekształceniu dwóch pierwszych wzorów i podstawieniu wszystkich wymaganych wartości dostajemy : 
%\bigbreak
%$\bm{I_2 = \frac{V_1 - I\cdot(R_4 + R_3)}{R_2} = \frac{5V - 0,00669A\cdot(100\Omega + 510\Omega)}{220\Omega} \approx  4,177mA}$
%\bigbreak
%$\bm{I_1 = \frac{V_1 - I\cdot(R_4 + R_3)}{R_1} = \frac{5V - 0,00669A\cdot(100\Omega + 510\Omega)}{360\Omega} \approx 2,55mA}$

\section{Zadanie 7.}
--------------------------------
%Zestawienie danych otrzymanych w wyniku obliczeń z danymi pomiarowymi.
%
%\begin{center}
%\begin{tabular}{|c||c|c|}
%\hline
%\textbf{Lp.} & \boldsymbol{$I_{Rx}$}(z tw. Thevenina) & \boldsymbol{$I_{Rx}$}(z obliczeń)\\
%\hhline{|=#=|=|}
%1. & $2,62mA$ & $2,55mA$\\
%\hline 
%2. & $4,26mA$ & $4,177mA$\\
%\hline
%\end{tabular}
%\end{center}

\section{Wnioski}
------------------------------
%
%Wartości natężenia prądów otrzymane w wyniku pomiarów i zastosowania twierdzenia Thevenina nieznacznie różnią się od wartości obliczonych analitycznie, 
%co jest najpewniej skutkiem:
%\begin{itemize}
%\item błędów pomiarowych używanej aparatury;
%\item ograniczonej precyzji wykonania rezystorów;
%\item niezerowego oporu płytki prototypowej i kabli użytych do pomiaru.
%\end{itemize}
%
%\begin{flushleft}
%Mimo tych różnic, otrzymane doświadczalnie wyniki dobrze odpowiadają wyliczonym wartościom (błąd nie przekracza 3\%), co
%stanowi o zasadności twierdzenia Thevenina.
%\end{flushleft}


\bibliography{IEEEabrv,refs}

\begin{thebibliography}{9}

\bibitem{thevenin}
  W trakcie przeprowadzania doświadczeń i pisania sprawozdania zespół korzystał głównie z materiałów ze strony http://mariusznaumowicz.ddns.net/materialy.html oraz z wiedzy własnej.

\end{thebibliography}

\end{document}
