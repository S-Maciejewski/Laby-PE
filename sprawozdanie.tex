\documentclass[polish,a4paper]{article}
\usepackage[T1]{fontenc}
\usepackage[utf8]{inputenc}
\usepackage{babel}
\usepackage{pslatex}
\usepackage{pgfplots}
\usepackage{circuitikz} 
\usetikzlibrary{circuits.ee.IEC}
\usepackage{anysize}
\marginsize{2.5cm}{2.5cm}{3cm}{3cm}

%makro do indeksów w tabeli
\newcommand{\PRzFieldDsc}[1]{\sffamily\bfseries\scriptsize #1}

%makro do informacji w tabeli
\newcommand{\PRzFieldCnt}[1]{\itshape #1}

%potężne makro tworzące tabelę z informacjami o teamie
\newcommand{\PRzHeading}[8]{
%% #1 - nazwa laboratorium
%% #2 - kierunek 
%% #3 - specjalność 
%% #4 - rok studiów 
%% #5 - symbol grupy lab.
%% #6 - temat 
%% #7 - numer lab.
%% #8 - skład grupy ćwiczeniowej

\begin{center}
\begin{tabular}{ p{0.32\textwidth} p{0.15\textwidth} p{0.15\textwidth} p{0.12\textwidth} p{0.12\textwidth} }

  &   &   &   &   \\
\hline
\multicolumn{5}{|c|}{}\\[-1ex]
\multicolumn{5}{|c|}{{\LARGE #1}}\\
\multicolumn{5}{|c|}{}\\[-1ex]

\hline
\multicolumn{1}{|l|}{\PRzFieldDsc{Kierunek}}	& \multicolumn{1}{|l|}{\PRzFieldDsc{Specjalność}}	& \multicolumn{1}{|l|}{\PRzFieldDsc{Rok studiów}}	& \multicolumn{2}{|l|}{\PRzFieldDsc{Symbol grupy lab.}} \\
\multicolumn{1}{|c|}{\PRzFieldCnt{#2}}		& \multicolumn{1}{|c|}{\PRzFieldCnt{#3}}		& \multicolumn{1}{|c|}{\PRzFieldCnt{#4}}		& \multicolumn{2}{|c|}{\PRzFieldCnt{#5}} \\

\hline
\multicolumn{4}{|l|}{\PRzFieldDsc{Temat Laboratorium}}		& \multicolumn{1}{|l|}{\PRzFieldDsc{Numer lab.}} \\
\multicolumn{4}{|c|}{\PRzFieldCnt{#6}}				& \multicolumn{1}{|c|}{\PRzFieldCnt{#7}} \\

\hline
\multicolumn{5}{|l|}{\PRzFieldDsc{Skład grupy ćwiczeniowej oraz numery indeksów}}\\
\multicolumn{5}{|c|}{\PRzFieldCnt{#8}}\\

\hline
\multicolumn{3}{|l|}{\PRzFieldDsc{Uwagi}}	& \multicolumn{2}{|l|}{\PRzFieldDsc{Ocena}} \\
\multicolumn{3}{|c|}{\PRzFieldCnt{\ }}		& \multicolumn{2}{|c|}{\PRzFieldCnt{\ }} \\

\hline
\end{tabular}
\end{center}
}
%koniec potężnego makro do tabeli

\begin{document}

%stworzenie tabeli - miejsce na zmienianie danych w tabeli
%indeksy do uzupełnienia
\PRzHeading{Laboratorium Elektrotechniki}{Informatyka}{--}{I}{I1}{Wprowadzenie}{2}{Ewa Fengler(132219), Sebastian Maciejewski(?), Jan Techner(132332)}{}

%ZADANIA

\section{Cel}

%Wprowadzenie, opis zadania i określenie celów doświadczenia

Celem laboratorium jest zbadanie właściwości obwodu rezonansowego szeregowego przedstawionego na rysunku \ref{fig:rlc} dla sygnału wejściowego napięciowego o różnych wartościach częstotliwości wejściowej. Badania przeprowadzono dla wartości elementów wyznaczonych przez prowadzącego zajęcia. Wartości elementów dla badanego obwodu: R=1k$\Omega$, L=66mH, C=10nF.

%rysunek obwodu - EWA
\tikzset{circuit declare symbol = AC source}
\tikzset{AC source IEC graphic/.style={
    circuit symbol lines,
    circuit symbol size=width 2 height 2,
    shape=generic circle IEC,
    /pgf/generic circle IEC/before background={
    \pgfpathmoveto{\pgfpoint{-0.8pt}{0pt}}
    \pgfpathsine{\pgfpoint{0.4pt}{0.4pt}}
    \pgfpathcosine{\pgfpoint{0.4pt}{-0.4pt}}
    \pgfpathsine{\pgfpoint{0.4pt}{-0.4pt}}
    \pgfpathcosine{\pgfpoint{0.4pt}{0.4pt}}
    \pgfusepath{stroke}
    },
    transform shape, draw
  }
}
\tikzset{circuit ee IEC/.append style=
  {set AC source graphic = AC source IEC graphic}
}
\begin{figure}[!h]

\centering
\begin{tikzpicture}[
    circuit ee IEC,
    x = 3cm, y = 2cm,
    every info/.style = {font = \scriptsize},
    set diode graphic = var diode IEC graphic,
    set make contact graphic = var make contact IEC graphic,
  ]

\draw (0,-1) to [vco] (0,1) --
	  (0,1) to [capacitor={farad=10n,info'={$C$}}] (1,1) --   
      (1,1) to [inductor={henry=66m,info'={$L$}}] (2,1) --
      (2,1) to [resistor={ohm=1k,info'={$R$}}] (2,-1) --
      (2,-1) to (0,-1);

\end{tikzpicture}
\caption{Badany obwód RLC}
\label{fig:rlc}
\end{figure}


\section{Pomiary}
Dla obwodu z rysunku \ref{fig:rlc} dokonano serią pomiarów napięcia na elementach R, L, C dla częstotliwości wejściowej z zakresu <0;12kHz>. Pomiary wykonane podczas badań zapisano w poniższej tabeli \ref{my-label}.

\begin{table}[!h]
\centering
\begin{tabular}{llll}
\hline
Częstotliwość(kHz) & $V_c$(mV) & $V_l$(mV) & $V_r$(mV) \\ \hline
0.8                & 727    & 9      & 29     \\
2.0                & 770    & 57     & 77     \\
...                & ...    & ...    & ...    \\ \hline
\end{tabular}
\caption{Wartości pomiarów na elementach obwodu dla różnych częstotliwości}
\label{my-label}
\end{table}

Zależności z tabeli \ref{my-label} przedstawiono jako charakterystykę napięciowo-częstotliwościową na rysunku \ref{fig:wyk}.
\begin{figure}[!h]
\centering
\begin{tikzpicture}[scale=1.0]
\begin{axis}[
xlabel={Częstotliwość [kHz]},
ylabel={Napięcie [mV]},
xmin=0,xmax=15,
ymin=0,ymax=2000,
legend pos=north west,
ymajorgrids=true,grid style=dashed
]

\addplot[color=red,mark=*]
coordinates {
(0.8,727)
(2.0,770)
(3.2,823)
(3.8,926)
(4.5,985)
(5.0,1134)
(5.8,1233)
(6.0,1348)
(6.2,1405)
(6.4,1466)
(6.6,1513)
};

\addplot[color=blue,mark=square]
coordinates {
(0.8,9)
(2.0,57)
(3.8,117)
(4.5,238)
(5.0,311)
(5.8,490)
(6.0,621)
(6.2,785)
(6.4,876)
(6.6,984)
};

\addplot[color=green,mark=triangle]
coordinates {
(0.8,29)
(2.0,77)
(3.8,114)
(4.5,169)
(5.0,209)
(5.8,271)
(6.0,329)
(6.2,371)
(6.4,403)
(6.6,427)
};

\legend{C,L,R}
\end{axis}
\end{tikzpicture}
\caption{Zależność napięć na elementach obwodu względem częstotliwości}
\label{fig:wyk}
\end{figure}

\section{Wnioski}
Zakładając poprawność przeprowadzonych badań jesteśmy w stanie stwierdzić, że dla częstotliwości równej... wg znanej nam wiedzy \cite{bolkowski1986teoria} zachodzi zjawisko rezonansu szeregowego, które pokrywa się z wartością obliczoną na podstawie wzorów algebraicznych \cite{bolkowski1986teoria}. Wszelkie rozbieżności mogą wynikać z...
\bibliographystyle{IEEEtran}

\bibliography{IEEEabrv,refs}

\end{document}

