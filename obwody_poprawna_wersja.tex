\documentclass[10pt,a4paper]{article}
\usepackage[T1]{fontenc}
\usepackage[utf8]{inputenc}
\usepackage[polish]{babel}
\usepackage{amsmath}
\usepackage[european]{circuitikz}
\usepackage{amsfonts}
\author{Jan Techner}
\begin{document}

Zadanie B, część I
\begin{figure}[!h]
\centering
\begin{circuitikz}[scale=1.1, font = \scriptsize]
\draw (0,0) to [R, l=$R_7$, a=1$\Omega$] (2,0) -- (2, 0.7) to [R, l=$R_5$, a=100$\Omega$] (4,0.7) -- (4,0)
(2,0) -- (2,-0.7) to [R, l=$R_6$, a=200$\Omega$] (4,-0.7) --  (4,0) -- (4.5,0) -- (4.5,2.1) to [R, l=$R_1$, a=2k$\Omega$] (6.5,2.1) -- (6.5,-2.1) to [R, l_=$R_4$, a^=270$\Omega$] (4.5,-2.1) -- (4.5,0)
(4.5,0.7) to [R, l=$R_2$, a=3k$\Omega$] (6.5,0.7)
(4.5,-0.7) to [R, l=$R_3$, a=1k$\Omega$] (6.5,-0.7)
(6.5,0) -- (7,0) -- (7,0.7) to [R, l=$R_8$, a=1$\Omega$] (9, 0.7) -- (9, -0.7) to [R, l_=$R_9$, a^=100$\Omega$] (7,-0.7) -- (7,0)
(9,0) -- (10,0);
\draw (-0.2,0) node {A}
	  (10.2,0) node {B};  
\end{circuitikz}
\end{figure}
\newline

Zadanie B, część II, obwód 1
\begin{figure}[!h]
\centering
\begin{circuitikz}[scale=1.1, font = \scriptsize]
\draw (0,0) -- (0.5,0) -- (0.5,0.7) to [R, l=$R_2$, a=2k$\Omega$] (2.5,0.7) to [R, l=$R_3$, a=2k$\Omega$] (4.5, 0.7) -- (4.5,0)
	  (0.5,0) -- (0.5,-0.7) to [R, l=$R_1$, a=1k$\Omega$] (4.5,-0.7) -- (4.5,0) -- (5,0); 
\path (-0.2,0) node {A}
	  (5.2,0) node {B};
\end{circuitikz}
\end{figure}
\newline

Zadanie B, część II, obwód 2
\begin{figure}[!h]
\centering
\begin{circuitikz}[scale=1.1, font = \scriptsize]
\draw (0,0) -- (0.5,0) -- (0.5,-0.7) to [R, l=$R_4$, a=100$\Omega$] (7,-0.7) -- (7, 0)
	  (0.5,0) -- (0.5,1.4) -- (1,1.4) -- (1,2.1) to [R, l=$R_1$, a=1k$\Omega$] (3,2.1) -- (3,0.7)
	  (1,1.4) -- (1,0.7) to [R, l=$R_2$, a=2k$\Omega$] (3,0.7) -- (3,1.4) to [R, l=$R_3$, a=1k$\Omega$] (5,1.4) to [R, l=$R_5$, a=100$\Omega$] (7,1.4) -- (7,0)-- (7.5,0);
\draw (-0.2,0) node {A}
	  (7.7,0) node {B};
\end{circuitikz}
\end{figure}
\newpage

Zadanie B, część II, obwód 3
\begin{figure}[!h]
\centering
\begin{circuitikz}[scale=1.1, font = \scriptsize]
\draw (0,0) -- (1,0) to [R, l=$R_1$, a=2k$\Omega$] (1,-2) -- (0, -2)
	  (1,0) -- (2,0) to [R, l=$R_3$, a=100$\Omega$] (4,0)
	  (2,0) -- (2,-1.4) to [R, l=$R_4$, a=1k$\Omega$] (4,-1.4) -- (4,0) -- (5,0) to [R, l=$R_2$, a=2k$\Omega$] (5, -2) -- (6,-2)
	  (5,0) -- (6,0);
\path (-0.2,0) node {A}
	  (-0.2,-2) node {B};
\end{circuitikz}
\end{figure}
\newline

Zadanie B, część II, obwód 4
\begin{figure}[!h]
\centering
\begin{circuitikz}[scale=1.1, font = \scriptsize]
\draw (0,0) -- (1,0) -- (1,-0.7) to [R, l=$R_4$, a=1k$\Omega$] (7.5,-0.7) -- (7.5, 0)
	  (1,0) -- (1,1.4) -- (1.5,1.4) -- (1.5,2.1) to [R, l=$R_5$, a=100$\Omega$] (3.5,2.1) to [R, l=$R_1$, a=1k$\Omega$] (5.5,2.1) -- (5.5,1.4)
	  (1.5,1.4) -- (1.5,0.7) to [R, l=$R_2$, a=2k$\Omega$] (5.5,0.7) -- (5.5,1.4)
	  (5.5,1.4) to [R, l=$R_3$, a=2k$\Omega$] (7.5,1.4) -- (7.5,0)--(8.5,0);
\path (-0.2,0) node {A}
	  (8.7,0) node {B};
\end{circuitikz}
\end{figure}
\newline

Zadanie B, część II, obwód 5
\begin{figure}[!h]
\centering
\begin{circuitikz}[scale=1.1, font = \scriptsize]
\draw (0,0) -- (1,0) to [R, l=$R_1$, a=2k$\Omega$] (1,-2) -- (0, -2)
	  (1,0) -- (2,0) to [R, l=$R_3$, a=2k$\Omega$] (4,0)
	  (2,0) -- (2,-1.2) to [R, l=$R_4$, a=1k$\Omega$] (4,-1.2) -- (4,0) -- (5,0) to [R, l=$R_2$, a=2k$\Omega$] (5, -2) -- (6,-2)
	  (5,0) -- (6,0)
	  (1,-2) -- (5, -2);
\path (-0.2,0) node {A}
	  (-0.2,-2) node {B};
\end{circuitikz}
\end{figure}
\newline


Zadanie B, część II, obwód 6
\begin{figure}[!h]
\centering
\begin{circuitikz}[scale=1.1, font = \scriptsize]
\draw (0,0) -- (0.5,0) to [R, l=$R_3$, a=1k$\Omega$] (2.5,0) -- (2.5, -0.7)
	  (0,-1.4) -- (0.5,-1.4) to [R, l=$R_4$, a=2k$\Omega$] (2.5,-1.4)--(2.5,-0.7) -- (3,-0.7) -- (3,0) to [R, l=$R_1$, a=100$\Omega$] (5,0) --(5,-0.7)
	  (3,-0.7) -- (3,-1.4) to [R, l=$R_2$, a=2k$\Omega$] (5,-1.4)-- (5,-0.7) -- (5.5, -0.7)
	  (0.5, -1.4) to (0.5, -2.8) to [R, l=$R_5$, a=1k$\Omega$] (5.5,-2.8)-- (5.5,-0.7);
\path (-0.2,0) node {A}
	  (-0.2,-1.4) node {B};
\end{circuitikz}
\end{figure}
\newpage

Zadanie C, część II
\begin{figure}[!h]
\centering
\begin{circuitikz}[scale=1.1, font = \scriptsize]

\draw (0,2) to [american voltage source, l_=$U_{in}$] (0,-2)
	  (0,2) -- (1.4, 2) to [R, l=$R_1$] (1.4, 0) to [R, l=$R_2$] (1.4,-2) -- (0,-2) 
	  (1.4,0) -- (2.8,0) -- (2.8,-0.75)
	  (2.8,-1.25) -- (2.8,-2) -- (1.4,-2);
\draw (2.8,-1) node {$U_{out}$};
\end{circuitikz}
\end{figure}
\newline

Zadanie C, część III
\begin{figure}[!h]
\centering
\begin{circuitikz}[scale=1.1, font = \scriptsize]
\draw (0,1) to [american voltage source, l_=$V_1$, a^=$5V$] (0,-1)
	  (0,1) to [R, l=$R_1$, a=2100$\Omega$] (3,1) -- (3,-1) -- (0,-1);
\end{circuitikz}
\end{figure}
\newline



Zadanie C, część IV, obwód 1
\begin{figure}[!h]
\centering
\begin{circuitikz}[scale=1.1, font = \scriptsize]
\draw (0,1) to [american voltage source, l_=$V_1$, a^=$5V$] (0,-1)
	  (0,1) to [R, l=$R_4$, a=1k$\Omega$] (3,1) to [R, l=$R_2$, a=2k$\Omega$] (5,1) -- (5,-1)
	  (3,-1) to [R, l=$R_3$, a=200$\Omega$] (3,1)
	  (0,-1) to [R, l=$R_1$, a=100$\Omega$] (3,-1) -- (5,-1);
\end{circuitikz}
\end{figure}


\end{document}